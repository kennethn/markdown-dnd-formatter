\documentclass[10pt,twocolumn]{article}

% =========================
% 1. Packages
% =========================

% Font and encoding
\usepackage{microtype}
\usepackage{soul}            % for underlines
\usepackage[normalem]{ulem}  % prevents soul/ulem conflicts
\usepackage{fontawesome5}
\usepackage[T1]{fontenc}
\usepackage{fontspec}
\usepackage[document]{ragged2e}
\usepackage[none]{hyphenat}
\usepackage{parskip}
\usepackage{needspace}
\usepackage{amssymb}

% Hyperlinks (no colors)
\usepackage[hidelinks]{hyperref}
\usepackage{xcolor}

% Page layout
\usepackage[paper=letterpaper,top=0.35in,bottom=0.50in,left=0.35in,right=0.35in]{geometry}
\usepackage{titlesec}
\usepackage{etoolbox}
\usepackage[all]{nowidow}

% Page numbering and headers/footers
\usepackage{fancyhdr}

% Lists
\usepackage{enumitem}

% Math
\usepackage{amsfonts}

% Tables
\usepackage{longtable}
\usepackage{booktabs}
\usepackage{array}
\usepackage{multirow}
\usepackage{colortbl}

% Colored boxes
\usepackage[most]{tcolorbox}

% Graphics
\usepackage{graphicx}

% =========================
% 2. Global Settings
% =========================

\RaggedRight
\setlength{\parskip}{8pt} % flexible ~20pt spacing
\setlength{\parindent}{0pt} % no indent for first lines
\setlength{\columnsep}{24pt}

\pagestyle{plain}
\raggedbottom

% =========================
% 3. Fonts and Font Commands
% =========================

\setmainfont{XCharter}[
  Ligatures=TeX,
  Numbers=Lining, 
  Mapping=tex-text,
  StylisticSet=0, % disables all stylistic sets
  Alternate=0
]                 
\newfontfamily\XCharterSC{XCharter}[
  Ligatures=TeX,
  Numbers=Lining, 
  Mapping=tex-text,
  Letters=SmallCaps,
  StylisticSet=0, % disables all stylistic sets
  Alternate=0 % disables character alternates
]          
\renewcommand{\textsc}[1]{\XCharterSC#1}

\newfontfamily\blockquoteFont{Ysabeau Office}[
  Ligatures=TeX,
  Numbers=Lining, 
  Mapping=tex-text,
  BoldFont    = {Ysabeau Office SemiBold},
  StylisticSet=0, % disables all stylistic sets
  Alternate=0 % disables character alternates
]   
\newfontfamily\headerfontbold{Ysabeau SC Bold}[
  Ligatures=TeX,
  Numbers=Lining,
  Scale=1.05,
  StylisticSet=0, % disables all stylistic sets
  Alternate=0 % disa
]       
\newfontfamily\headerfont{Ysabeau SC}[
  Ligatures=TeX,
  Numbers=Lining,
  Scale=1.05,
  StylisticSet=0, % disables all stylistic sets
  Alternate=0 % disa
]
\newfontfamily\monsterFont{Ysabeau Office}[
  Ligatures=TeX,
  Numbers=Lining, 
  BoldFont    = {Ysabeau Office SemiBold},
  Mapping=tex-text               
]

\newfontfamily\emojifont{Symbola}

% =========================
% 4. Section Formatting
% =========================

\newcommand{\stickysection}{\needspace{5\baselineskip}}
\newcommand{\stickysubsection}{\needspace{4\baselineskip}}
\newcommand{\stickysubsubsection}{\needspace{3\baselineskip}}

\newcommand{\sectionsize}{\LARGE}
\newcommand{\subsectionsize}{\Large}
\newcommand{\subsubsectionsize}{\large}

\renewcommand{\ULdepth}{0.5ex}  % adjust this value to move the underline
\renewcommand{\ULthickness}{0.03em}  % underline thickness

\linespread{1.2}

% =========================
% 5. Colors
% =========================

\definecolor{sectioncolor}{HTML}{B85042}      
\definecolor{subsectioncolor}{HTML}{B85042}    
\definecolor{subsubsectioncolor}{HTML}{B85042} 
\definecolor{subsubsubsectioncolor}{HTML}{B85042} 

\definecolor{keywordcolor}{HTML}{B85042}
\definecolor{highlightcolor}{HTML}{E7E8D1}
\definecolor{encountercolor}{HTML}{dfaca5}
\definecolor{imagecolor}{HTML}{A7BEAE}

% =========================
% 6. Text and List Styling
% =========================

\let\oldtextbf\textbf
\renewcommand{\textbf}[1]{\oldtextbf{{#1}}}

% Stylize bullets in itemize lists
\setlist[itemize,1]{label=\textcolor{keywordcolor}{\textbullet}}
\setlist[itemize,2]{label=\textbullet}
\setlist[itemize,3]{label=\textcolor{keywordcolor}{\textendash}}

% Always override item spacing for itemize outside monsterblocks
\setlist[itemize]{itemsep=4pt, topsep=0pt, parsep=0pt, partopsep=0pt}

% Ensure Pandoc-generated \tightlist honors this too
\providecommand{\tightlist}{
  \setlength{\itemsep}{4pt}
  \setlength{\topsep}{0pt}
  \setlength{\parsep}{0pt}
  \setlength{\parskip}{0pt}
  \setlength{\partopsep}{0pt}
}

% =========================
% 7. Section Titles and Spacing
% =========================

\titleformat{\section}[block]{\stickysection\sectionsize\color{sectioncolor}\headerfontbold}{}
  {0pt}{}
\titleformat{\subsection}[block]
  {\stickysubsection\subsectionsize\color{subsectioncolor}\headerfontbold}
  {}
  {0pt}
  {}
  [\vspace{1pt}\color{subsectioncolor}\hrule height 1pt]

\titleformat{\subsubsection}[block]
  {\subsubsectionsize\color{subsubsectioncolor}\headerfontbold}
  {}
  {0pt}
  {\stickysubsubsection\subsubsectionsize\color{subsubsectioncolor}\headerfont}
  [\vspace{1pt}\color{subsubsectioncolor}\hrule height 1pt]
\titlespacing*{\section}{0pt}{14pt plus 2pt minus 1pt}{10pt}
\titlespacing*{\subsection}{0pt}{12pt plus 1pt minus 1pt}{10pt}
\titlespacing*{\subsubsection}{0pt}{10pt plus 1pt minus 1pt}{12pt}
\titlespacing*{\subsubsubsection}{0pt}{10pt plus 1pt minus 1pt}{12pt}

% =========================
% 8. Headers, Footers, and Pagination
% =========================

\fancyhf{}
\fancyfoot[C]{\thepage}
\renewcommand{\headrulewidth}{0pt}
\pagestyle{fancy}
\setlength{\footskip}{15pt}

\clubpenalty=9996
\widowpenalty=9999
\brokenpenalty=4991
\predisplaypenalty=10000
\postdisplaypenalty=1549
\displaywidowpenalty=1602

% =========================
% 9. Environments and Boxes
% =========================

% Graphics
\setkeys{Gin}{width=0.48\textwidth, keepaspectratio}
\providecommand{\maxwidth}{\linewidth}
\providecommand{\maxheight}{0.9\textheight}
\providecommand{\pandocbounded}[1]{#1}

\renewenvironment{quote}
  {%
    \begingroup
      % restore normal indent & spacing
      \setlength{\parindent}{1em}%
      \setlength{\parskip}{0pt}%
      \begin{tcolorbox}[myquote,
        before upper={%
          \let\textbf\oldtextbf
          % <-- set indent & skip *inside* the box
          \setlength{\parindent}{1.5em}%
          \setlength{\parskip}{0pt}%
          \noindent              % <-- suppress *first* paragraph indent
        }%
      ]%
  }
  {%
      \end{tcolorbox}%
    \endgroup
  }

% Blockquote style (D&D-style box with Sans Serif)
\tcbset{myquote/.style={
  breakable,
  enhanced,
  colback={highlightcolor},
  boxrule=0pt,
  arc=0pt,
  left=6pt,
  right=6pt,
  top=4pt,
  bottom=4pt,
  boxsep=4pt,
  before skip=12pt,
  after skip=12pt,
  borderline west={1.5pt}{0pt}{keywordcolor},
  borderline east={1.5pt}{0pt}{keywordcolor},
  sharp corners,
  fontupper=\blockquoteFont,
  before upper={%
    \setlength{\parskip}{4pt}%
    \setlength{\baselineskip}{10pt}%
  }
}}

% =========================
% 10. Miscellaneous Tweaks
% =========================

\makeatletter
% Remove vertical space before \section at top of page
\patchcmd{\@startsection}
  {\@afterindenttrue}
  {\@afterindentfalse}
  {}{}
\patchcmd{\@startsection}
  {\@afterheading}
  {\vspace*{-\topskip}\@afterheading}
  {}{}

% Suppress \parskip within list environments
\patchcmd{\@itemize}{\parskip\z@}{\parskip=0pt}{}{}
\patchcmd{\@enumerate}{\parskip\z@}{\parskip=0pt}{}{}
\patchcmd{\@description}{\parskip\z@}{\parskip=0pt}{}{}
\makeatother

% =========================
% 11. Document Content
% =========================

\begin{document}

\begingroup
\makeatletter
\setlength{\@fptop}{0pt}
\vspace*{-\topskip}
\makeatother
\endgroup
\vspace*{-\topskip}

\begingroup
\begin{center}
\fontsize{36pt}{36pt}\color{highlightcolor}\selectfont
\faDiceD20
\\
\LARGE\color{sectioncolor}\headerfontbold
Ep. 20: 2025–08–16\\[-18pt]
\color{sectioncolor}\rule{\linewidth}{2pt}
\\[-2pt]
\end{center}
\endgroup

\section{Starting Information}\label{starting-information}

\begin{itemize}
\tightlist
\item
  \textbf{Location:} Obsidian Bog
\item
  \textbf{Date:}~Uktar 20 in Toril
\end{itemize}

\section{Review the Characters}\label{review-the-characters}

\begin{itemize}
\tightlist
\item
  \textbf{\textcolor{keywordcolor}{\textbf{\textsc{glynda}}}.}~(Summer)
  Dwarf paladin. Will have 3rd level spells she's eager to try.~
\item
  \textbf{\textcolor{keywordcolor}{\textbf{\textsc{kelmenor}}}.}~(Patrick)
  Elven rogue. Additional sneak attack damage.~
\item
  \textbf{\textcolor{keywordcolor}{\textbf{\textsc{ryland}}}.}~(Nick)
  Human wizard. Will have 5th level spells.~
\item
  Advance to 10th level after the Harmonic of Stillnight is acquired.
\end{itemize}

\section{Run of Show}\label{run-of-show}

\begin{itemize}
\tightlist
\item[$\square$]
  The party enters the Obsidian Bog looking for the Dead Man's Cross
\item[$\square$]
  They row through an old battlefield and are attacked by ghast
  gravecallers and a corpse flower
\item[$\square$]
  While camping they might encounter a halfling tinker
\item[$\square$]
  At Dead Man's Cross, they determine the location of the Tidekeeper's
  Crypt is at Fadewell Rocks
\item[$\square$]
  Possible return to Gloomwrought before their sea travel to obtain
  items
\item[$\square$]
  While sailing to Fadewell Rocks they have an encounter with a galleon
  manned by undead
\item[$\square$]
  Arriving at Fadewell Rocks, they will need to enter and exit the crypt
  before the tide returns
\item[$\square$]
  In the Tidekeeper's Crypt are a puzzle, shadow eel, crabs, several
  undead, and a Boneclaw
\item[$\square$]
  The party obtains the Harmonic of Stillnight
\item[$\square$]
  While sailing back from Fadewell Rocks, they have an encounter with an
  aboleth
\end{itemize}

\section{Secrets and Clues}\label{secrets-and-clues}

\begin{itemize}
\tightlist
\item[$\square$]
  The Tidecaller's Crypt lies in the Fadewell Rocks and the Harmonic of
  Stillnight is kept there
\item[$\square$]
  The bard who is buried in the crypt co-created the Harmonic with a fey
  bard who is buried in the Feywild with its twin
\end{itemize}

\section{Strong Start}\label{strong-start}

\begin{tcolorbox}[
  colback={imagecolor},
  coltext=black,
  colframe=black,
  boxrule=1pt,
  arc=6pt,
  left=4pt,
  right=4pt,
  top=2pt,
  bottom=2pt,
  boxsep=4pt,
  before skip=10pt,
  after skip=10pt,
  fontupper={\blockquoteFont\small\linespread{0.9}\selectfont\color{black}}
]

\faPhotoVideo\hspace{0.8em}\begin{minipage}[t]{\dimexpr\linewidth-1.8em\hangindent=1.8em\hangafter=0}Rowing
through the bog

\end{minipage}\end{tcolorbox}

\begin{tcolorbox}[
  enhanced,
  breakable,
  colback={encountercolor},
  colframe=black,
  boxrule=1pt,
  coltext=black,
  arc=6pt,
  left=4pt,
  right=4pt,
  top=2pt,
  bottom=2pt,
  boxsep=4pt,
  before skip=10pt,
  after skip=10pt,
  fontupper={\blockquoteFont\small\linespread{0.9}\selectfont\color{black}}
]

\faSkull\hspace{0.8em}\begin{minipage}[t]{\dimexpr\linewidth-1.8em\hangindent=1.8em\hangafter=0}A
burst of laughter echoes from the mist.~It sounds~\emph{exactly}~like
one of the PCs. They didn't laugh.

\end{minipage}\end{tcolorbox}

\begin{tcolorbox}[
  enhanced,
  breakable,
  colback={encountercolor},
  colframe=black,
  boxrule=1pt,
  coltext=black,
  arc=6pt,
  left=4pt,
  right=4pt,
  top=2pt,
  bottom=2pt,
  boxsep=4pt,
  before skip=10pt,
  after skip=10pt,
  fontupper={\blockquoteFont\small\linespread{0.9}\selectfont\color{black}}
]

\faSkull\hspace{0.8em}\begin{minipage}[t]{\dimexpr\linewidth-1.8em\hangindent=1.8em\hangafter=0}The
swamp expels~the ruins of an old building, complete with~skeletons,~a
grisly reminder that this place was once~inhabited.

\end{minipage}\end{tcolorbox}

\begin{quote}
The muck seems to only be waist deep here. The boat occasionally grinds
across the mud. As your boat glides forward, the mist clings to the
water like cold breath. First the stench hits you.~

Next there's a thump against the bow of the boat. Something is in the
swamp. You can only see shapes, unmoving humps in the reeds, broken
spears jutting from the muck, a snapped standard fluttering from a
rotted pole. The remains of a battlefield.

Then you finally see them. Dozens of them.~Bloated bodies~drift among
the cattails and black water like grotesque water lilies. Some are
little more than armor-wrapped bones. Others are swollen, split,
half-submerged, gazing upward with eyeless sockets or slack jaws still
locked mid-scream.~A drowned, bloated horse floats in the marshy water,
its mouth pulled back in a rictus grin.~Crows and ravens feast upon the
bodies of the fallen. The birds peck and tear at the dead; blood stains
their feathers and beaks, giving them a somewhat infernal appearance.
Preoccupied with their meals, they seem unconcerned with your presence.

A rstone cairn stands atop a low hill emerging from the swamp. A ragged,
bloody banner hanging from a pole thrust deep into the cairn stirs
listlessly in the faint breeze. The banner bears the symbol of a skull
with the horned head of a goat.

Up ahead, at the very edge of your vision you can see a shield
half-protruding from the muck, reflecting the light from your lantern.
Unlike the other objects in view, it looks clean and pristine.
\end{quote}

\begin{tcolorbox}[
  colback={imagecolor},
  coltext=black,
  colframe=black,
  boxrule=1pt,
  arc=6pt,
  left=4pt,
  right=4pt,
  top=2pt,
  bottom=2pt,
  boxsep=4pt,
  before skip=10pt,
  after skip=10pt,
  fontupper={\blockquoteFont\small\linespread{0.9}\selectfont\color{black}}
]

\faPhotoVideo\hspace{0.8em}\begin{minipage}[t]{\dimexpr\linewidth-1.8em\hangindent=1.8em\hangafter=0}Bog
Battlefield

\end{minipage}\end{tcolorbox}

\begin{quote}
The oar dips into the black water with a slurp. Reeds sway, drowned
weapons poke up like warning signs. Silence hangs over the bog like a
held breath. Then---

The water erupts.~On~both sides~of the boat, hands shoot upward: long,
clawed, and covered in mossy black flesh. Figures rise,~not swimming,
but climbing, like the water itself is holding them up.

Its eyes are~milky but aware, scanning the boat like it's selecting
tools for dissection. They look like ghasts, but there is a much greater
intelligence here, these are not your typical undead.

Its jaw opens wider than it should, revealing rows of blackened teeth,
and it speaks in a voice that echoes like a funeral hymn in a flooded
church:~``You brought life to this place,'' it hisses, voice like
drowned silk. ``Let's fix that.''
\end{quote}

\begin{tcolorbox}[
  colback={imagecolor},
  coltext=black,
  colframe=black,
  boxrule=1pt,
  arc=6pt,
  left=4pt,
  right=4pt,
  top=2pt,
  bottom=2pt,
  boxsep=4pt,
  before skip=10pt,
  after skip=10pt,
  fontupper={\blockquoteFont\small\linespread{0.9}\selectfont\color{black}}
]

\faPhotoVideo\hspace{0.8em}\begin{minipage}[t]{\dimexpr\linewidth-1.8em\hangindent=1.8em\hangafter=0}Ghast
Gravecaller Image

\end{minipage}\end{tcolorbox}

\begin{tcolorbox}[
  enhanced,
  breakable,
  colback={encountercolor},
  colframe=black,
  boxrule=1pt,
  coltext=black,
  arc=6pt,
  left=4pt,
  right=4pt,
  top=2pt,
  bottom=2pt,
  boxsep=4pt,
  before skip=10pt,
  after skip=10pt,
  fontupper={\blockquoteFont\small\linespread{0.9}\selectfont\color{black}}
]

\faSkull\hspace{0.8em}\begin{minipage}[t]{\dimexpr\linewidth-1.8em\hangindent=1.8em\hangafter=0}\textbf{Ghast
Gravecaller (CR 6)} -- \emph{AC} 16, \emph{HP} 97, \emph{Speed} 30 ft,
\emph{Atk} +7 (2d10+4), \emph{Saves} CON +5, WIS +5, \emph{Immune}:
Necrotic, Poison, Poisoned, \emph{Abilities}: Horrific Necrosis, Stench,
Swamp Grasp, Spellcasting

\end{minipage}\end{tcolorbox}

If the PCs are winning, the Corpse Flower will attack. Otherwise wait
until the encounter is going the PCs' way.

\begin{quote}
More movement.

A shape lurches upright from beneath the pile, taller than any man. This
is something much bigger. It isn't one body. It's~many. Twisted
together. You see limbs tangled like roots, jaws yawning from between
ribs, a snapped spine flapping like a whip from its back.
Flowers---vile, fungal, and reeking---burst from its flesh in sickly
yellow tufts.

It shifts toward you, dragging a corpse with it like a child with a
doll.

The air turns~\emph{sweet}~with decay. And then it starts to~\emph{move
fast.}
\end{quote}

\begin{tcolorbox}[
  enhanced,
  breakable,
  colback={encountercolor},
  colframe=black,
  boxrule=1pt,
  coltext=black,
  arc=6pt,
  left=4pt,
  right=4pt,
  top=2pt,
  bottom=2pt,
  boxsep=4pt,
  before skip=10pt,
  after skip=10pt,
  fontupper={\blockquoteFont\small\linespread{0.9}\selectfont\color{black}}
]

\faSkull\hspace{0.8em}\begin{minipage}[t]{\dimexpr\linewidth-1.8em\hangindent=1.8em\hangafter=0}\textbf{Corpse
Flower (CR 8)} -- \emph{AC} 12, \emph{HP} 168, \emph{Speed} 20 ft, climb
20 ft, \emph{Atk} +5 (2d4+2), \emph{Saves} CON +9, WIS +4,
\emph{Resist}: Cold, Lightning, Necrotic, nonmagical damage;
\emph{Abilities}: Digest Corpse, Spawn Zombie, Stench of Death

\end{minipage}\end{tcolorbox}

\begin{tcolorbox}[
  colback={imagecolor},
  coltext=black,
  colframe=black,
  boxrule=1pt,
  arc=6pt,
  left=4pt,
  right=4pt,
  top=2pt,
  bottom=2pt,
  boxsep=4pt,
  before skip=10pt,
  after skip=10pt,
  fontupper={\blockquoteFont\small\linespread{0.9}\selectfont\color{black}}
]

\faPhotoVideo\hspace{0.8em}\begin{minipage}[t]{\dimexpr\linewidth-1.8em\hangindent=1.8em\hangafter=0}Corpse
Flower Attacks

\end{minipage}\end{tcolorbox}

\subsubsection{After Combat
Investigation}\label{after-combat-investigation}

Several magic items can be found in the muck
(see~\textcolor{keywordcolor}{\textbf{\textsc{magic item rewards}}}).

Several combatants seem to have wore obsidian raven masks, many now lie
shattered in the mud, or float eerily just beneath the surface. Others
seem to have worn~shields and helms with images of skulls and the horned
head of a goat.~\textbf{See Lore section below.}

\subsubsection{Getting to Dead Man's
Cross}\label{getting-to-dead-mans-cross}

The characters will need to travel overland six miles after they
progress as far as possible in the boat. The sisters will be reluctant
to leave the boat and insist on staying with it.

\section{Potential Scenes}\label{potential-scenes}

\subsection{Back to Gloomwrought}\label{back-to-gloomwrought}

If the party is seeking magical items:

\begin{quote}
It is considered against the will of the
\textcolor{keywordcolor}{\textbf{\textsc{raven queen}}} to cheat death
in such a way. To drown is to honor her. Only one here that I know of
who risks her offense and deals in such things is~Knex of the Veiled
League.
\end{quote}

\textbf{Knex} is part of the Veiled League. If the party finds Knex:

\begin{quote}
Were I to be able to obtain such a thing, and I am not saying I can, I
would sell them to you for 550 GP each. Yes, gold. I know from whence
you came and what it is worth to you. I take on a great risk with the
Raven Queen's faithful in even discussing such items.
\end{quote}

If the party seeks information about Therynzhaal's Domain or the
\textbf{Domains of Dread}:

\begin{quote}
The topic is cloaked in fear, rumor, and contradiction. They're places
where the mists take you, and you don't come back.~They say the mists
punish those who meddle too much. Legends say each domain is a prison,
trapping its evil Darklord forever. Nobody knows who imprisons them,
there are references to ``Dark Forces'' and ``The Dark Powers.'' Sailors
speak of ships venturing too far into the Stormy Sea and getting lost in
the mist, never to return.

In order to enter the mists to reach a domain, one must have a talisman.
Usually something that belongs or belonged to the Darklord.

Several people mention the Vistani people, and their ability to travel
the mists and come and go as they please. Legend says these people were
originally from the Domains of Dread. There are stories of Vistani
guides leading visitors into domains.
\end{quote}

\textbf{DC 13 History} check:

\begin{quote}
Vistani were usually regarded with suspicion because of their ``Evil
Eye'', the ability to curse anyone they looked at.~Vistani traveled in
tight-knit family groups or tribes using barrel-topped
wagons\href{https://forgottenrealms.fandom.com/wiki/Vistani\#cite_note-CoS-p26--1}{{[}1{]}}~that
they referred to as vardos.~Some Vistani would offer shelter, food, and
protection for travelers and strangers alike.~Vistani were known for
their musical talent as well as their skill in silversmithing,
coppersmithing, cooking, weaving, entertaining, storytelling,
toolmaking, and horse-trading.~The Vistani had darker complexions and
usually sported jet-black hair. They wore colorful and cheery clothing
and were big drinkers.
\end{quote}

If the party meets \textbf{Jalparna} the Vistani:

\begin{quote}
Each is a prison built by its prisoner. The bars are forged from regret.
The warden is always watching---and it's always themselves.~There are
many domains, each ruled by a Darklord, but not all are known by name.
Some are whispered only among the Vistani.~

\textcolor{keywordcolor}{\textbf{\textsc{therynzhaal}}}. Yes, I know
that name, although I have never met anyone who has traveled his domain.
My people call him the Knight of the Hollow Gleam, or the Weeping Flame,
and imply he's a new Darklord, still forging his prison.~He was not born
of cruelty. But now cruelty wears his face.

Crossing into one is easy. Leaving?~That depends on~what brought you
there in the first place. The Mists won't open for force or faith. But
if you carry a piece of what was lost, they'll part like water.~Do not
speak his name in the mists. There are echoes there that remember how to
answer.
\end{quote}

\subsection{Death from the deep}\label{death-from-the-deep}

\begin{tcolorbox}[
  enhanced,
  breakable,
  colback={encountercolor},
  colframe=black,
  boxrule=1pt,
  coltext=black,
  arc=6pt,
  left=4pt,
  right=4pt,
  top=2pt,
  bottom=2pt,
  boxsep=4pt,
  before skip=10pt,
  after skip=10pt,
  fontupper={\blockquoteFont\small\linespread{0.9}\selectfont\color{black}}
]

\faSkull\hspace{0.8em}\begin{minipage}[t]{\dimexpr\linewidth-1.8em\hangindent=1.8em\hangafter=0}\textbf{Aboleth
(CR 10)} -- \emph{AC} 17, \emph{HP} 240, \emph{Speed} 10 ft, swim 40 ft,
\emph{Atk} +9 (2d6+5), \emph{Saves} CON +6, INT +8, WIS +6,
\emph{Skills} Perception +10, \emph{Resist}: 3×Legendary Resist,
\emph{Abilities}: Dominate Mind, Consume Memories, Mucus Cloud, Probing
Telepathy

\end{minipage}\end{tcolorbox}

\begin{quote}
Suddenly, the waters around the boat collapse and the boat tilts
precariously. Just as you reach to grab something solid the waters surge
upward in an explosive wave.
\end{quote}

Everyone must make a \textbf{DC 12 Dexterity}~saving throw or be thrown
into the water and knocked prone.

\begin{quote}
A massive, greenish eel-like creature with lamprey teeth bears down on
your vessel.~Two long tentacles sprout from its head. Set above the
triangular, spherical nose are three eyes, each one set atop the other.
Tendrils and a few shorter tentacles dangle from the bottom of the head.
A thick gray mucus covers the creature and floats away on the surface of
the water like an oil slick.
\end{quote}

The aboleth will use its telepathic summons against a random passenger
and then attack.

\subsection{Fishing ship encounter}\label{fishing-ship-encounter}

\begin{quote}
The seas have calmed a bit and you regain your footing. Zahira shouts,
``ship ahead.'' Just at the edge of the darkness on your starboard side
you can make out a carrack with three masts, towering over you. On the
side of the ship you see the name \emph{The Happy Squid}. Sailors in
dark coats and long-brimmed hats move effortlessly across the deck. A
massive ballista on the ship's bow is turns, tilts, and aims directly at
you.~
\end{quote}

The Happy Squid is captained by~\textbf{Crumskorn}, a gnome monster
hunter. Sarcastic and witty, he is the only living soul onboard, his
crew consists entirely of undead. The majority are skeletons, his first
mate is an orc ghoul named \textbf{Muk} and his cook is a headless
zombie called \textbf{Cookie}.~

Despite the macabre appearance, Crumskorn is friendly. He does not see
the PCs as a threat and it has been a long time since he had living
companionship. He welcomes them onboard for rest, water, and a meal. He
will be extremely curious to get information about Gloomwrought, it's
clear it has been a long time since he has spent any time ashore. He
describes himself as a ``fisherman'' although in actuality he is a
monster hunter.~

After the party dines on seafood, he will show them his catch: eight
\textbf{Churnborn (chuul)}. They are a horrible mix of crustacean,
insect, and serpent. These tall yellow-green lobster-like creatures have
four long legs, two large claws, and a mass of tentacles protruding from
their mouths.~

\begin{itemize}
\tightlist
\item
  A \textbf{DC 13 Arcana} check identifies the creatures as chuul,
  primeval sentient crustaceans. Their tentacles are paralyzing.
\end{itemize}

He warns the PCs to beware of the churnborn's masters,~\textbf{Ancestors
of the Deep, (aboleth)}~ancient 20 foot long fish-like creatures that
have roamed these seas ``a million years before the gods were born.'' He
pleas with them not to underestimate their intelligence and to ``guard
your minds.''~

\begin{itemize}
\tightlist
\item
  A \textbf{DC 15 Arcana} check recognizes the description of an
  aboleth, the ancient creators of the chuul.
\end{itemize}

\subsection{Camping in the wilderness}\label{camping-in-the-wilderness}

\begin{tcolorbox}[
  colback={imagecolor},
  coltext=black,
  colframe=black,
  boxrule=1pt,
  arc=6pt,
  left=4pt,
  right=4pt,
  top=2pt,
  bottom=2pt,
  boxsep=4pt,
  before skip=10pt,
  after skip=10pt,
  fontupper={\blockquoteFont\small\linespread{0.9}\selectfont\color{black}}
]

\faPhotoVideo\hspace{0.8em}\begin{minipage}[t]{\dimexpr\linewidth-1.8em\hangindent=1.8em\hangafter=0}Swamp
Campsite

\end{minipage}\end{tcolorbox}

\textbf{What is on your character's mind as you sit in front of the
fire? What story do you tell your companions?}

\begin{quote}
As you warm yourself against the fire, you hear a metallic clacking
sound interspersed with the ringing of bells. The sound gets louder,
accompanied by trudging footsteps.

``Greetings, there! Might you make room for company beside your fire? I
have goods to trade for your warmth.''

A man steps into view at the edge of the camp light. Half-elven by
appearance, he bears a ridiculously large pack stacked with items tied
to every surface.
\end{quote}

The man is~\textbf{Zopher Zyznana}~(NG male half-elf): Zopher can be
heard by the clacking of pots and the ringing of bells that hang from
his overlarge pack. A tinker and trader, the half-elf is an excellent
conversationalist and lively camp-mate. He peddles a range of mundane
but unusual goods, from witching bells to shrunken heads to glass dolls,
all the while blowing marvelous smoke shapes from his pipe. There is a
25\% chance he will be carrying each item of equipment from the Player's
Handbook the PCs seek. A 10\% chance for unusual or out-of-the-ordinary
items.

Zopher warns that the campsite is prone to flooding if it rains heavily
while the party camp here, they are in for an unpleasant surprise after
a few hours. He offers to help them move the camp about 50 feet uphill
and he pulls his fair share of labor.

He has limited information about the Shadowfell beyond his immediate
surroundings, and he's hazy about where he came from or how he got
there.

\section{Fantastic Locations}\label{fantastic-locations}

\subsection{Dead Man's Cross}\label{dead-mans-cross}

\begin{quote}
At the center of the crossroads stands a grisly marker, a signpost
adorned with an ancient corpse bound to its crosspieces. The constant
abrasion of windblown dust has scoured clean the placards of the
signpost and turned the body into a desiccated husk. The corpse's skin
is as black as night and as hard as boiled leather. The lids of its eyes
are stitched closed, and its lips have pulled back around its teeth into
a grin.
\end{quote}

\begin{tcolorbox}[
  colback={imagecolor},
  coltext=black,
  colframe=black,
  boxrule=1pt,
  arc=6pt,
  left=4pt,
  right=4pt,
  top=2pt,
  bottom=2pt,
  boxsep=4pt,
  before skip=10pt,
  after skip=10pt,
  fontupper={\blockquoteFont\small\linespread{0.9}\selectfont\color{black}}
]

\faPhotoVideo\hspace{0.8em}\begin{minipage}[t]{\dimexpr\linewidth-1.8em\hangindent=1.8em\hangafter=0}Dead
Man's Cross

\end{minipage}\end{tcolorbox}

\vspace{6pt}\noindent\stickysubsubsection

{\headerfont\normalsize \color{subsubsectioncolor}Where is the Tidekeeper’s Crypt?}

\par

\begin{quote}
His bony finger aligns with the mist like the needle of a compass, and
suddenly\ldots{} \emph{you} \emph{know}. It isn't sight. It isn't sound.
It's in your bones. A pressure behind your eyes. A sense that if you
step that way, you won't be lost. You have never been there, but you
feel as if you have. A particular cove in a rocky archipelago over the
sea.

Zahira looks at her sisters, and Nafira and Sahira, at the same moment,
say, ``Fadewell Rocks.'' Nafira explains, ``It's in a children's rhyme
in Gloomwrought. Little boys and girls sing it while skipping rope.''
\end{quote}

The rhyme is a call-and-response and it's usually accompanied by a
skipping and jumping game.

\begin{quote}
(Call)\\
Fadewell Rocks are cold and gray\\
(Response)\\
Singing Spires dance and sway\\
(Call)\\
One held dusk and one held dawn\\
(Response)\\
One stayed here, the other gone\\
(Call)\\
Tiptoe quiet, don't break time\\
(Response)\\
Feel the rhythm, hear the rhyme\\
(Call)\\
Two sang once, then slipped away\\
(Response)\\
Resting deep where echoes play
\end{quote}

\vspace{6pt}\noindent\stickysubsubsection

{\headerfont\normalsize \color{subsubsectioncolor}Where is Therynzhaal’s Domain?}

\par

\begin{quote}
His bony finger aligns point. Suddenly you have an image of a
tumultuous, choppy ocean disappearing into mist. You just know this is
the Stormy Sea, at a particular compass point. You've never navigated a
ship before, but you just know you can get there. The mist is calling
you.
\end{quote}

\vspace{6pt}\noindent\stickysubsubsection

{\headerfont\normalsize \color{subsubsectioncolor}Effects}

\par

After using Dead Man, must immediately draw
from~\textcolor{keywordcolor}{\textbf{\textsc{despair deck}}}~and the
Wisdom save is DC 15.

The Dead Man's directions work like the
\textcolor{keywordcolor}{\textbf{\textsc{find the path}}}~spell. For 30
days, as long as you are on the same plane of existence as the
destination, you know how far it is and in what direction it lies. While
you are traveling there, whenever you are presented with a choice of
paths along the way, you automatically determine which path is the
shortest and most direct route (but not necessarily the safest route) to
the destination.

\subsection{Fadewell Rocks}\label{fadewell-rocks}

\begin{tcolorbox}[
  colback={imagecolor},
  coltext=black,
  colframe=black,
  boxrule=1pt,
  arc=6pt,
  left=4pt,
  right=4pt,
  top=2pt,
  bottom=2pt,
  boxsep=4pt,
  before skip=10pt,
  after skip=10pt,
  fontupper={\blockquoteFont\small\linespread{0.9}\selectfont\color{black}}
]

\faPhotoVideo\hspace{0.8em}\begin{minipage}[t]{\dimexpr\linewidth-1.8em\hangindent=1.8em\hangafter=0}Fadewell
Rocks at high tide

\end{minipage}\end{tcolorbox}

At low tide: a muddy, cracked expanse reveals winding paths between the
stacks and a hidden cave entrance beneath one particularly twisted
formation.~The entrance (set at the base of a rock stack) is~0.6 miles
(about 3,000 ft) from the nearest water's edge at the lowest tide. The
entrance is partially sealed by a tide door which opens during high tide
and closes during low tide.

At high tide: the entire region is submerged under 50 feet of black,
brackish water. The tide rolls in fast and violently, often mistaken for
a supernatural force. (See
\textcolor{keywordcolor}{\textbf{\textsc{fadewell rocks tide chart}}}).
Due to this, it's extremely dangerous to keep a boat docked near the
rocks.

Barnacle scars show the high-water mark, which towers above head height

Describe large rocks and boulders strewn across the beach (can be used
to solve Under Pressure)

\subsection{Tidekeeper's Crypt}\label{tidekeepers-crypt}

\begin{tcolorbox}[
  colback={imagecolor},
  coltext=black,
  colframe=black,
  boxrule=1pt,
  arc=6pt,
  left=4pt,
  right=4pt,
  top=2pt,
  bottom=2pt,
  boxsep=4pt,
  before skip=10pt,
  after skip=10pt,
  fontupper={\blockquoteFont\small\linespread{0.9}\selectfont\color{black}}
]

\faPhotoVideo\hspace{0.8em}\begin{minipage}[t]{\dimexpr\linewidth-1.8em\hangindent=1.8em\hangafter=0}Fadewell
Rocks at low tide

\end{minipage}\end{tcolorbox}

\subsubsection{General Description}\label{general-description}

Constructed specifically to house the
\textcolor{keywordcolor}{\textbf{\textsc{harmonic of stillnight}}},~this
crypt was a marvel of planar engineering and arcane will. It lies buried
in the intertidal zone, accessible only during the narrow window of low
tide, a liminal place for a liminal relic.

\textbf{Kaerith the Duskcaller}, bard of the Umbral Assembly and
co-creator of the Harmonic, chose to be entombed here as its eternal
guardian. Powerful abjurations and geomantic rituals once held the sea
at bay during its construction. The crypt was meant to remain protected,
disturbed only at the destined hour, by those in true need.

But centuries passed. And something else found its way in:~a
power-hungry necromancer, obsessed with finding Kaerith's resting place
and claiming the Harmonic for himself, shattered the wards. He brought
with him apprentices, sacrifices, and all the arrogance of one who
believed he could outwit time and tide.~The sea returned faster than he
anticipated, sealing him inside. Trapped and dying, with air running
out, he attempted a desperate lichdom ritual, but failed. The phylactery
shattered. His soul, twisted by obsession and fear, lashed itself into
undeath.~He became a \textbf{Boneclaw}, a malformed echo of the power he
sought.

His dead apprentices rose again at his command. Others who braved the
crypt after him --- adventurers, tomb robbers, scholars --- found
themselves drowned and turned into his servants. Now, they wander the
dark in varying stages of soggy rot and madness.

\pandocbounded{\includegraphics[keepaspectratio]{Files/Crypt map.png}}

\subsubsection{Tracking Time}\label{tracking-time}

The characters will have about 3 hrs 15 min to get in and out, maximum.
Place glass beads in a bowl with each bead representing five minutes of
in-game time (36 beads: 30 white, 9 red). Take beads out of the bowl
every time the characters expend time.

\subsubsection{Underwater Combat}\label{underwater-combat}

A fight underwater follows these rules.

\vspace{6pt}\noindent\stickysubsubsection

{\headerfont\normalsize \color{subsubsectioncolor}Impeded Weapons}

\par

When making a melee attack roll with a weapon underwater, a creature
that lacks a Swim Speed has Disadvantage on the attack roll unless the
weapon deals Piercing damage. A ranged attack roll with a weapon
underwater automatically misses a target beyond the weapon's normal
range, and the attack roll has Disadvantage against a target within
normal range.

\vspace{6pt}\noindent\stickysubsubsection

{\headerfont\normalsize \color{subsubsectioncolor}Fire Resistance}

\par

Anything underwater has Resistance to Fire damage (explained in ``Damage
and Healing'').

\vspace{6pt}\noindent\stickysubsubsection

{\headerfont\normalsize \color{subsubsectioncolor}Spells}

\par

Spells can be cast underwater as long as they don't use verbal
components. Verbal components can be used if the caster has the ability
to breathe underwater.~

\subsubsection{Difficult Terrain}\label{difficult-terrain}

Ceilings are 15 feet high unless stated otherwise. The slick,
algae-covered floors are \textbf{difficult terrain}. Movement is halved
and all Dexterity rolls are at disadvantage. The Dash action requires a
\textbf{DC 12 Acrobatics} (Dexterity) check.

\subsubsection{Doors}\label{doors}

There are six tide-controlled doors in the crypt using impressive
engineering. Unlike a typical tide gate, these work in reverse: they
open when the tide is high and close when the tide recedes. Floats
attached to the ocean side of the door cause the doors to open with the
rising waters. As the tide recedes, soft-close pneumatic governors in
the hinges slow the door's closing to allow the waters to fully recede
(the floats settle at a 30 minute delay to the water level).

The end result is a flooded crypt during high tide with the doors open
but a dry crypt during low tide with the doors closed. Opening a closed
door from the outside (ocean-side) requires a \textbf{DC 13 Athletics
(Strength)} check. Opening a closed door from the inside (crypt-side)
requires a \textbf{DC 17 Athletics (Strength)} check. The doors can be
propped open with an object large enough to block the door from closing.
The doors cannot be closed below the water level of the floats.

\begin{tcolorbox}[
  colback={imagecolor},
  coltext=black,
  colframe=black,
  boxrule=1pt,
  arc=6pt,
  left=4pt,
  right=4pt,
  top=2pt,
  bottom=2pt,
  boxsep=4pt,
  before skip=10pt,
  after skip=10pt,
  fontupper={\blockquoteFont\small\linespread{0.9}\selectfont\color{black}}
]

\faPhotoVideo\hspace{0.8em}\begin{minipage}[t]{\dimexpr\linewidth-1.8em\hangindent=1.8em\hangafter=0}Tide
Door

\end{minipage}\end{tcolorbox}

\pandocbounded{\includegraphics[keepaspectratio]{Files/Tide_Door.png}}

\subsubsection{1: Cave Entrance}\label{cave-entrance}

\textbf{\textnormal{\emojifont\char"2B50} What are the PCs using for
light?}

\begin{quote}
A dark, jagged mouth leads to a gradually upward sloping cavernous
corridor about 5 feet wide. The walls and floors of the corridor are
covered with barnacles, algae, and seaweed. Seawater is still rushing
past your feet as you peer into the dark cave.
\end{quote}

A \textbf{DC 14 Survival} check will learn that the corridors gently
slope up, and appear to have been designed intentionally to allow the
crypt to drain.

\subsubsection{1a: Tide Door}\label{a-tide-door}

\begin{quote}
This corridor ends at a round metal doorway, really a hatch. The door
looks well-engineered and seems to open upwards with unusual
double-jointed hinges. The door is partially open about eighteen inches
and water is receding through the open gap. Three bulbous oblate
spheroids made of polished metal hang off rods attached to the middle of
the door and are currently sitting at water level. As you watch, you can
see the door's hinges softly closing.
\end{quote}

A \textbf{DC 15 Investigation} check will reveal the information above
under \textbf{Doors.}

\subsubsection{2: Outer Cavern}\label{outer-cavern}

\begin{quote}
The chamber is soaking wet and smells like the ocean. Algae, snails, and
seaweed hang from every surface. A few fish flop on the ground, stranded
as the tide receded. Water drips and streams from the ceiling and runs
down every inch of the walls.~

That's not all though, three giant crabs turn toward you while a
disgusting eel-like creature, seemingly amphibious, tilts its head and
begins slithering your way.
\end{quote}

\begin{tcolorbox}[
  enhanced,
  breakable,
  colback={encountercolor},
  colframe=black,
  boxrule=1pt,
  coltext=black,
  arc=6pt,
  left=4pt,
  right=4pt,
  top=2pt,
  bottom=2pt,
  boxsep=4pt,
  before skip=10pt,
  after skip=10pt,
  fontupper={\blockquoteFont\small\linespread{0.9}\selectfont\color{black}}
]

\faSkull\hspace{0.8em}\begin{minipage}[t]{\dimexpr\linewidth-1.8em\hangindent=1.8em\hangafter=0}\textbf{3x}~\textbf{Giant
Crab (CR 1/8)} -- \emph{AC} 15, \emph{HP} 24, \emph{Speed} 30 ft, swim
30 ft, \emph{Atk} +3 (1d6+1), \emph{Skills} Stealth +3, \emph{Traits}:
Amphibious, Grapple on hit (DC 11);~\textbf{Giant Shadow Eel (CR 2)} --
\emph{AC} 12, \emph{HP} 104, \emph{Speed} 30 ft, swim 30 ft, \emph{Atk}
+6 (2d6+4), \emph{Saves} STR +4, \emph{Traits}: Constrict (DC 14),
Amphibious, Blindsight 10 ft

\end{minipage}\end{tcolorbox}

Three~\textbf{giant crabs}~and a \textbf{giant sea eel} (giant
constrictor snake stat block)~make their lairs in this cavern (easy
challenge).

\subsubsection{3: Inner Cavern}\label{inner-cavern}

\begin{tcolorbox}[
  enhanced,
  breakable,
  colback={encountercolor},
  colframe=black,
  boxrule=1pt,
  coltext=black,
  arc=6pt,
  left=4pt,
  right=4pt,
  top=2pt,
  bottom=2pt,
  boxsep=4pt,
  before skip=10pt,
  after skip=10pt,
  fontupper={\blockquoteFont\small\linespread{0.9}\selectfont\color{black}}
]

\faSkull\hspace{0.8em}\begin{minipage}[t]{\dimexpr\linewidth-1.8em\hangindent=1.8em\hangafter=0}\textbf{3x}
\textbf{Drowned Wight (CR 3)} -- \emph{AC} 15, \emph{HP} 45,
\emph{Speed} 30 ft, \emph{Atk} +4 (1d8+2 + 1d6 cold), \emph{Skills}
Perception +3, Stealth +4, \emph{Traits}: Cold Aura (1d6), Grapple on
hit (DC 12), Life Drain (bonus, 1d8+2 necrotic, max HP
reduction);~\textbf{6x}~\textbf{Zombie, Cavefish (CR 1/2)} -- \emph{AC}
10, \emph{HP} 37, \emph{Speed} 20 ft, swim 40 ft, \emph{Atk} +4 (1d6+2),
\emph{Saves} WIS +0, \emph{Immune}: Poison, \emph{Traits}: Undead
Fortitude, Stench (DC 10)

\end{minipage}\end{tcolorbox}

\begin{quote}
As you press deeper into the half-submerged chamber, the air thickens
with the scent of salt rot and mildew. Water drips incessantly from
stone above, echoing like a heartbeat in the gloom. The tide-slick floor
is littered with barnacled bones and strands of kelp dragged in from the
sea.~

Then\ldots{} they come into view. Six pale and rubbery human-like
zombies writhe like drowned nightmaresTheir flesh hangs in strips,
stretched translucent and pocked with bloated sacs of brine and gas.
Algae webs cling to their torsos like sickly veils. One zombie has no
jaw. Another's torso is fully opened, wrinkly, waterlogged viscera
hanging at its feet. They have seemingly adapted to their underwater
environment.~Most of their fingers and toes are webbed and fleshy fins
run down the spine and legs of others, poking through stretches of dead
flesh. An overpowering stench of rot surrounds them. Every footstep
squishes. Every breath is a struggle not to gag.~

Looming behind them come three drowned wights, taller and crueler. Their
faces are etched with the silent fury of those who died choking on
seawater and bile. Barnacles crust across their chests like armor. One
of the creature's face is horridly decomposed; muscles and sinew can
easily be made out through the ruin of its rotting cheeks.

A faint glow emanates from their wicked swords and bows. Their eyes too
glow faintly blue, a flickering candle in the darkness. With every step
toward you these creatures leak saltwater from their mouths and joints,
the tide slowly draining.
\end{quote}

\subsubsection{4: Crypt Foyer}\label{crypt-foyer}

\begin{quote}
The tide door groans as it opens, grinding against wet stone.~The
chamber beyond is dark and unnaturally still.~

At the far end stands a massive stone sarcophagus. A wooden trapdoor is
just barely visible in the ceiling.
\end{quote}

\begin{tcolorbox}[
  enhanced,
  breakable,
  colback={encountercolor},
  colframe=black,
  boxrule=1pt,
  coltext=black,
  arc=6pt,
  left=4pt,
  right=4pt,
  top=2pt,
  bottom=2pt,
  boxsep=4pt,
  before skip=10pt,
  after skip=10pt,
  fontupper={\blockquoteFont\small\linespread{0.9}\selectfont\color{black}}
]

\faSkull\hspace{0.8em}\begin{minipage}[t]{\dimexpr\linewidth-1.8em\hangindent=1.8em\hangafter=0}\textbf{Boneclaw
(CR 12)} -- \emph{AC} 16, \emph{HP} 175,~\emph{Speed} 40 ft, \emph{Atk}
+8 (3d10+4+2d8), \emph{Saves} DEX +7, WIS +6, \emph{Resist}: Cold,
Necrotic; \emph{Immune}: Poison, Charmed, Frightened, etc.,
\emph{Abilities}: Reaping Claws, Deadly Reach, Shadow Stealth

\end{minipage}\end{tcolorbox}

A \textbf{boneclaw} is~here. The boneclaw will use its Shadow Stealth to
remain unseen, then try to attack with surprise after the party has
entered the room. It uses its 15 foot reach and Shadow Jump to attack
while staying out of melee reach. If the party moves to retreat, it will
use Shadow Jump to flank them.

\begin{tcolorbox}[
  colback={imagecolor},
  coltext=black,
  colframe=black,
  boxrule=1pt,
  arc=6pt,
  left=4pt,
  right=4pt,
  top=2pt,
  bottom=2pt,
  boxsep=4pt,
  before skip=10pt,
  after skip=10pt,
  fontupper={\blockquoteFont\small\linespread{0.9}\selectfont\color{black}}
]

\faPhotoVideo\hspace{0.8em}\begin{minipage}[t]{\dimexpr\linewidth-1.8em\hangindent=1.8em\hangafter=0}Boneclaw

\end{minipage}\end{tcolorbox}

\begin{quote}
A towering and skeletal shape must have been hidden in the shadows. Its
face is a horrifying visage of hatred. Its eyes burn with a cold,
calculating malice. Its arms end in comically long fingers longer than
its torso. At the end of its fingers are wicked, sharp claws.

It's skeletal jaw opens, it's head tilts, and from deep somewhere inside
comes a voice. ``You're already too late.''
\end{quote}

\subsubsection{5: Burial Room}\label{burial-room}

\begin{quote}
As you scramble up the narrow shaft you notice a high tide water mark
where the stone, although damp, appears to be untouched by the sea. You
peer into a room that is moist, but obviously dry throughout the day. A
large sarcophagus sits in the middle of the chamber.

The walls are decorated with colorful frescoes depicting two women in
various scenes:speaking to one another, playing instruments and singing,
crafting something at a forge, and lying on their deathbeds.

You can see a heavy stone door in the middle of one wall. The door seems
to be set precisely into the surrounding walls, completely flush against
its edges, with not even the slightest gap visible. On the floor in
front of the sarcophagus is a large diamond shape mosaic approximately 5
feet long composed of small colored tiles. The diamond is outlined in
orange, with a red triangle in its center. The tip of the triangle
points to the singular stone door.~
\end{quote}

\begin{tcolorbox}[
  enhanced,
  breakable,
  colback={encountercolor},
  colframe=black,
  boxrule=1pt,
  coltext=black,
  arc=6pt,
  left=4pt,
  right=4pt,
  top=2pt,
  bottom=2pt,
  boxsep=4pt,
  before skip=10pt,
  after skip=10pt,
  fontupper={\blockquoteFont\small\linespread{0.9}\selectfont\color{black}}
]

\faSkull\hspace{0.8em}\begin{minipage}[t]{\dimexpr\linewidth-1.8em\hangindent=1.8em\hangafter=0}Under
Pressure puzzle protects the Harmonic of Stillnight

\end{minipage}\end{tcolorbox}

This 15x15 is reachable through the ceiling of Room \#4. It always stays
dry because the high water mark is just below the floor. However, there
is not enough air to breathe for more than a few hours.

Bas relief frescoes and illustrations fill the walls, accompanied by
writing in Sylvan, tell the story of the two bards, their coming
together to bring balance to the Feywild and the Shadowfell, the
creation of the Stillnight, their deaths, and the mirrored crypts in
which they are buried. Their names are:~\textbf{Kaerith the
Duskcaller,}~Bard of the Umbral Assembly of the Shadowfell,
and~\textbf{Ivenne the Tideweaver,} Chordwright of the Seelie Court of
the Feywild. Their crypts are labeled \textbf{Fadewell Rocks} and
\textbf{Singing Spires}, respectively. Kaerith is often depicted with a
raven above her head and Ivenne with a hummingbird.

The body inside the sarcophagus is well-preserved, showing no signs of
tampering or vandalism.~The skeleton clutches the desiccated stalks of a
bunch of colorful flowers, surprisingly dry despite the humid
conditions. You recognize these flowers from your visit to the Feywild.

\textbf{Treasure:} A small silver pin depicting a tuning fork adorns her
chest. She is buried in a fine leather coat tooled with elven script
(250 gp), a well-crafted porcelain cup adorned with gold depicting a
raven and hummingbird in a circle chasing each other's tails (250 gp), a
lacquered wooden coffer inlaid with ornate electrum scrollwork (250 gp),
a velvet vest threaded with electrum (250 gp), and an electrum cup
engraved with noble imagery (250 gp), an ebony harp (1,500 gp), and a
musical pipe made of blackwood (500 gp).

\subsubsection{Puzzle: Under Pressure}\label{puzzle-under-pressure}

\begin{tcolorbox}[
  colback={imagecolor},
  coltext=black,
  colframe=black,
  boxrule=1pt,
  arc=6pt,
  left=4pt,
  right=4pt,
  top=2pt,
  bottom=2pt,
  boxsep=4pt,
  before skip=10pt,
  after skip=10pt,
  fontupper={\blockquoteFont\small\linespread{0.9}\selectfont\color{black}}
]

\faPhotoVideo\hspace{0.8em}\begin{minipage}[t]{\dimexpr\linewidth-1.8em\hangindent=1.8em\hangafter=0}Diamond
Floor Mosaic

\end{minipage}\end{tcolorbox}

\begin{itemize}
\tightlist
\item
  \textbf{Door:}~The door does not move when pushed, pulled, or
  otherwise manually forced. The door itself is not magical. Behind the
  door is a 5x5 foot chamber with an obsidian pedestal. Upon the
  pedestal lies the~\textbf{Harmonic of Stillnight}.
\item
  \textbf{Floor:}~The floor of this room works as a magical scale. It
  determines the weight of each member of the party as they enter and
  calculates the combined weight of the group. It transmits this
  information to the diamond pressure plate which magically calibrates
  itself accordingly.
\item
  \textbf{Diamond Pressure Plate:}~A~\textbf{DC 12 Perception}~check
  notices that the diamond has a small groove along the perimeter. The
  plate opens the door fully only with the combined weight of the entire
  party.~Stepping off slams it shut instantly. The entire floor, with
  the exception of the diamond, emits an aura
  of~\textbf{Divination}~magic. The diamond~emits an aura
  of~\textbf{Transmutation}.
\item
  \textbf{Extra Challenge:}~Door stays open only while full party weight
  remains. The item they're retrieving (fragile relic) cannot be
  teleported without destroying it.
\item
  \textbf{Environmental Pressure:}~Tide rapidly rising---visible
  countdown recommended (e.g., sand timer or clock at table). Players
  must hurry.
\item
  \textbf{Solutions:}

  \begin{itemize}
  \tightlist
  \item
    \textbf{Environmental:}~Move heavy objects onto the plate. Rocks
    from the beach or broken off the Fadewell pillars. It will take two
    round trips to bring enough stones up. It takes 15 minutes to get
    down to the beach. Roll on the table below to determine how long it
    takes to bring back each load. A clever party will only bring back
    enough stones to match the weight of a single member. Therefore, two
    or more characters can return with enough stones in a single trip.
  \item
    \textbf{Physical (Hard):}~Dash through rapidly closing door (DC 33
    Athletics or Acrobatics check; punishing but epic!).
  \item
    \textbf{Magic}. Fey Step works but door will slam shut.~
  \end{itemize}
\item
  \textbf{Hints:}~Encourage environmental solutions or creative teamwork
  under intense time pressure.
\end{itemize}

\subsubsection{Gathering Stones}\label{gathering-stones}

A~\textbf{DC 13 Wisdom (Survival)}~check is required for a PC to find
enough portable stones to match their own bodyweight. Every failed check
uses 10 minutes, a natural 20 on the roll halves this time and a natural
1 doubles it

\subsubsection{Carrying Stones}\label{carrying-stones}

A~\textbf{Strength (Athletics)}~check is required to haul the stones
back to the room. Roll for each trip (two are required). The amount of
time a trip takes depends on the success of the roll. Characters can
decide they are pressing to the edge of their fitness. In exchange, they
gain +10 to their roll and one level of Exhaustion.

\begin{center}
{\sffamily\fontsize{8pt}{8pt}\selectfont
\rowcolors{2}{highlightcolor}{white}
\begin{tabular}{ll}
\toprule
\textbf{Avg. Strength Check} & \textbf{Time to Haul Back} \\
\midrule
25+ & 15 min \\
18--24 & 20 min \\
12–17 & 25 min (base) \\
Below 12 & 30 min \\
\bottomrule
\end{tabular}}
\end{center}

\subsubsection{6. False Crypt}\label{false-crypt}

\begin{quote}
This room contains a large stone sarcophagus. There is another tide door
on the far wall and a wooden trapdoor in the ceiling.
\end{quote}

This room is a false crypt, designed to be similar to Room 4. The sole
purpose of this room is to waste visitors' time before the tide rolls
back in.

\begin{itemize}
\tightlist
\item
  \textbf{Ceiling Trapdoor:} The ceiling trapdoor is trapped. If it's
  pulled down, a \textbf{DC 15 Dexterity} save is required to avoid
  taking \textbf{4d10 damage} from a massive rusty barnacle-encrusted
  anchor that will drop from the compartment above.

  \begin{itemize}
  \tightlist
  \item
    \textbf{DC 15~Passive Perception:} Notice water dripping steadily
    from the trapdoor, and faint rust flakes on the floor below.
  \item
    \textbf{DC 20~Investigation:} Spot the tension in the trapdoor's
    hinges and a slight downward bow in the wood.
  \end{itemize}
\item
  \textbf{Tomb:} The tomb is completely empty.
\item
  \textbf{False Tide Door:} The tide door on the eastern wall of the
  cavern is a fake.

  \begin{itemize}
  \tightlist
  \item
    \textbf{DC 15 Investigation:} Notices the hinges on this door are
    different from others they've seen and appear to be cosmetic.
    Tapping or knocking will also reveal the door is set against stone.
  \end{itemize}
\end{itemize}

\section{Important NPCs}\label{important-npcs}

\begin{itemize}
\tightlist
\item
  \textbf{Crumskorn.} A sarcastic gnome sea captain of The Happy Squid
  with an undead crew.
\item
  \textbf{Ivenne the Tideweaver.}~An archfey bard and chordwright of the
  Seelie Court, co-creator of the Harmonic of Stillnight. Buried in the
  Singing Spires in the Feywild.
\item
  \textbf{Jalparna.} A Vistani barrel maker who lives in a barrel-topped
  wagon in the Plaza District.
\item
  \textbf{Kaerith the Duskcaller.}~Bard of the Umbral Assembly and
  co-creator of the Harmonic of Stillnight. Buried in the Tidekeeper's
  Crypt.
\item
  \textbf{Knex.} Guy from the Veiled League who can acquire rare items.
\item
  \textbf{Nafira yr Rafiq al-Zarid.} Calishite captain of the Morning
  Gull (use \textcolor{keywordcolor}{\textbf{\textsc{pirate captain}}}).
\item
  \textbf{Samira.} Sailor, mute (use
  \textcolor{keywordcolor}{\textbf{\textsc{pirate}}})
\item
  \textbf{Zahira.} her sister, fellow sailor (use
  \textcolor{keywordcolor}{\textbf{\textsc{pirate}}})
\item
  \textbf{Zopher Zyznana}~(NG male half-elf). Tinker and trader.
\end{itemize}

\section{Magic Item Rewards}\label{magic-item-rewards}

\begin{itemize}
\tightlist
\item
  \textcolor{keywordcolor}{\textbf{\textsc{potion of healing}}}~(Superior)
  in a water-logged waist pouch
\item
  \textcolor{keywordcolor}{\textbf{\textsc{candle of invocation}}}(Linked
  to Abyss, Orcus) in a water-proofed steel case. There is Infernal
  script on the case and the candle itself.
\item
  \textcolor{keywordcolor}{\textbf{\textsc{shield of shield +1}}}~held
  by a skeletal soldier wearing half-disintegrated platemail, the shield
  itself is clean and unaffected by the elements
\item
  \textcolor{keywordcolor}{\textbf{\textsc{dagger of echoing silence +1}}}~protruding
  from the back of a floating corpse, the dagger makes no sound when
  removed or dropped
\item
  Maleficarum Nocturnum. A sinister spellbook bound in shadow-dyed
  leather, its pages are etched with silver ink that glows faintly in
  darkness. The book is damp but that pages are undamaged and the script
  is pristine. The book has the words~Maleficarum Nocturnum seared into
  the cover
\item
  \textcolor{keywordcolor}{\textbf{\textsc{harmonic of stillnight}}}~is
  the key to saving
  \textcolor{keywordcolor}{\textbf{\textsc{therynzhaal}}} and restoring
  him to silver dragon form
\end{itemize}

\subsubsection{Maleficarum Nocturnum}\label{maleficarum-nocturnum}

A sinister spellbook bound in shadow-dyed leather, its pages are etched
with silver ink that glows faintly in darkness.

\textbf{Spells Contained:}

1st Level

\begin{itemize}
\tightlist
\item
  \emph{Disguise Self}~(Illusion)
\item
  \emph{Cause Fear}~(Necromancy)
\end{itemize}

2nd Level

\begin{itemize}
\tightlist
\item
  \emph{Blindness/Deafness}~(Necromancy)
\item
  \emph{Mirror Image}~(Illusion)
\end{itemize}

3rd Level

\begin{itemize}
\tightlist
\item
  \emph{Animate Dead}~(Necromancy)
\item
  \emph{Fear}~(Necromancy)
\end{itemize}

4th Level

\begin{itemize}
\tightlist
\item
  \emph{Greater Invisibility}~(Illusion)
\item
  \emph{\textcolor{keywordcolor}{\textbf{\textsc{gloomveil}}}}~(Illusion,
  Homebrew)
\end{itemize}

5th Level

\begin{itemize}
\tightlist
\item
  \emph{Danse Macabre}~(Necromancy)
\item
  \emph{Cloudkill}~(Conjuration)
\end{itemize}

6th Level

\begin{itemize}
\tightlist
\item
  \emph{Circle of Death}~(Necromancy)
\end{itemize}

\section{\texorpdfstring{\textcolor{keywordcolor}{\textbf{\textsc{orcus}}}
and \textcolor{keywordcolor}{\textbf{\textsc{raven queen}}}
Lore}{ and  Lore}}\label{orcus-and-raven-queen-lore}

The horned goat is a symbol of the Demon Lord of Undeath,
\textcolor{keywordcolor}{\textbf{\textsc{orcus}}}. Those dead were
likely cultists of Orcus.~The masks and spiraling, mirrored iconography
as being associated with the Raven Queen.~

\begin{itemize}
\tightlist
\item
  \textbf{Arcana DC 10:} As with many extraplanar entities,
  \textcolor{keywordcolor}{\textbf{\textsc{orcus}}} is not killed if
  defeated on the material plane, merely banished back to his own corner
  of the abyss.
\item
  \textbf{Arcana DC 15:}
  \textcolor{keywordcolor}{\textbf{\textsc{orcus}}} wields a wand whose
  touch is death to the living, though it has been known to find its way
  into mortal hands for brief periods before being reclaimed.
\item
  \textbf{History DC 10:}~

  \begin{itemize}
  \tightlist
  \item
    The \textcolor{keywordcolor}{\textbf{\textsc{raven queen}}} is the
    most popularly worshipped deity in the Shadowfell, particularly
    amongst the Shadar-Kai. People pray to her for an easy death and
    transition to the spirit world. They ask her to protect their
    deceased loved ones from becoming undead. Over the years, she has
    essentially become a goddess of faith, and people ask her for good
    fortune and luck.
  \item
    The cult of \textcolor{keywordcolor}{\textbf{\textsc{orcus}}} is the
    least tolerated of the many faiths of the Shadowfell. Shadar-kai and
    most shadowborn consider worship of the demon prince to be a high
    offense. As a result, the cult of
    \textcolor{keywordcolor}{\textbf{\textsc{orcus}}} operates out of
    view, conducting their profane rites in secrecy.~~Beyond the cities
    and settlements, \textcolor{keywordcolor}{\textbf{\textsc{orcus}}}
    holds greater sway.
  \end{itemize}
\item
  \textbf{History DC 20:}
  \textcolor{keywordcolor}{\textbf{\textsc{orcus}}} was reportedly slain
  some years ago by a rival to his throne of undeath, and for a time
  wandered as a foul shade of himself, wreaking havoc upon the mortal
  and divine with the terrible powers he discovered in death before
  eventually being restored to life by a faithful servant.
\item
  \textbf{Religion DC 10:}~The soldiers appear to be Raven Knights, holy
  warriors pledged to the service of the
  \textcolor{keywordcolor}{\textbf{\textsc{raven queen}}}.
  \textcolor{keywordcolor}{\textbf{\textsc{raven queen}}} followers
  destroy undeath as acts of devotion, seeing it as a perversion of the
  journey of the soul through death.~Neither a mortal nor a god, neither
  living nor undead, the
  \textcolor{keywordcolor}{\textbf{\textsc{raven queen}}} is feared by
  all for her ability to collect memories and even souls.
\item
  \textbf{Religion DC 15:}~

  \begin{itemize}
  \tightlist
  \item
    \textcolor{keywordcolor}{\textbf{\textsc{orcus}}} is known as the
    Lord of Undeath, and worshipped by those that live in death or seek
    to. He reigns from his palace Everlost in Thanatos, the Tomb of the
    Abyss, a layer where death and undeath hunt the living without rest
    or mercy, even those of demonic nature themselves.
  \item
    The \textcolor{keywordcolor}{\textbf{\textsc{raven queen}}} rules
    from a fortress called the Palace of Memories (or the Fortress of
    Memories) in the Shadowfell. The elves tell the story of an elven
    queen from the Feywild more beloved by her people than the gods
    themselves. Pained by the eternal conflict between Corellon and
    Lolth, the queen channeled the souls and magic of her people to
    ascend to divine status, so that her pleas for peace may be heeded
    by the gods. But just as her ritual was about to succeed, something
    intervened that caused the queen to die. However, her soul did not
    travel to the City of the Dead. She and her followers were hurled
    through the Shadowfell, where the queen became the goddess of death.
  \end{itemize}
\item
  \textbf{Religion DC 20:}~This battle was likely part of a wider planar
  war. The shared history of
  \textcolor{keywordcolor}{\textbf{\textsc{orcus}}} and the
  \textcolor{keywordcolor}{\textbf{\textsc{raven queen}}} as beings who
  transcended death creates a spiritual rivalry. While
  \textcolor{keywordcolor}{\textbf{\textsc{orcus}}} is locked in eternal
  warfare with his rivals Graz't and Demogorgon, rumors abound that he
  has set his ambitions higher; at the very throne of the gods.
  \textcolor{keywordcolor}{\textbf{\textsc{orcus}}} has long sought to
  overthrow the \textcolor{keywordcolor}{\textbf{\textsc{raven queen}}}
  and claim dominion over the Shadowfell, allowing him to control the
  cycle of death.~
\end{itemize}

\clearpage

\vfill\break

\begingroup

\makeatletter
\clubpenalty=150
\widowpenalty=150
\displaywidowpenalty=150
\monsterFont
\fontsize{9pt}{10pt}\selectfont
\setlength{\parskip}{4pt}
\makeatletter
\setlist[itemize]{left=1.5em, itemsep=1pt, topsep=6pt, parsep=0pt, partopsep=0pt}
\setlist[enumerate]{left=1.5em, itemsep=2pt, topsep=6pt, parsep=0pt, partopsep=0pt}

\renewcommand{\sectionsize}{\Large}
\renewcommand{\subsectionsize}{\normalsize}
\renewcommand{\subsubsectionsize}{\normalsize}

\titlespacing*{\section}{0pt}{6pt}{4pt}
\titlespacing*{\subsection}{0pt}{6pt}{4pt}
\titlespacing*{\subsubsection}{0pt}{4pt}{4pt}
\titlespacing*{\subsubsubsection}{0pt}{4pt}{4pt}

\titleformat{\section}[block]
  {\stickysubsection\sectionsize\color{sectioncolor}\headerfontbold}
  {}
  {0pt}
  {}
  [\vspace{0pt}\color{sectioncolor}\hrule height 1pt]

\titleformat{\subsection}[block]
  {\stickysubsection\subsectionsize\color{subsectioncolor}\headerfont}
  {}
  {0pt}
  {}
  [\vspace{0pt}\color{sectioncolor}\hrule height 1pt]
\titleformat{\subsubsection}[block]
  {\stickysubsection\subsubsectionsize\color{subsubsectioncolor}\headerfont}
  {}
  {0pt}
  {}
  [\vspace{0pt}\color{sectioncolor}\hrule height 1pt]
% Local override for \tightlist so global version doesn't bleed in
\def\tightlist{%
  \setlength{\itemsep}{0pt}%
  \setlength{\topsep}{4pt}%
  \setlength{\parsep}{0pt}%
  \setlength{\parskip}{0pt}%
  \setlength{\partopsep}{0pt}%
}
\makeatother

\section{Aboleth}\label{aboleth}

\emph{Large}~\emph{Aberration}

\setlength{\itemsep}{0pt}

\begin{itemize}
\tightlist
\item
  \textbf{Armor Class:}~17
\item
  \textbf{Hit Points:}~171 (18d10+72)
\item
  \textbf{Speed:}~walk 10 ft. swim 40 ft.
\item
  \textbf{Challenge Rating:}~11 (7,200 XP)
\item
  \textbf{Source:}~A5e Monstrous Menagerie, page 16
\end{itemize}

\begin{center}
{\sffamily\fontsize{8pt}{8pt}\selectfont
\rowcolors{2}{highlightcolor}{white}
\begin{tabular}{llllll}
\toprule
\textbf{STR} & \textbf{DEX} & \textbf{CON} & \textbf{INT} & \textbf{WIS} & \textbf{CHA} \\
\midrule
20 (+5) & 12 (+1) & 18 (+4) & 20 (+5) & 20 (+5) & 18 (+4) \\
\bottomrule
\end{tabular}}
\end{center}

\setlength{\itemsep}{0pt}

\begin{itemize}
\tightlist
\item
  \textbf{Saving Throws}: Dex +5, Con +8, Int +9, Wis +9
\item
  \textbf{Skills:}~deception +8, history +9, stealth +5
\item
  \textbf{Senses:}~blindsight 30 ft., darkvision 120 ft., passive
  Perception 15
\item
  \textbf{Languages:}~Deep Speech, telepathy 120 ft.
\end{itemize}

\subsubsection{Special Abilities}\label{special-abilities}

\textbf{Amphibious:}~The aboleth can breathe air and water.

\textbf{Innate Spellcasting:}~The aboleths spellcasting ability is
Charisma (spell save DC 16). It can innately cast the following spells,
requiring no components: 3/day each: detect thoughts (range 120 ft,
desc: ), project image (range 1 mile), phantasmal force

\subsubsection{Actions}\label{actions}

\textbf{Multiattack:}~The aboleth attacks three times with its tentacle.

\textbf{Tentacle:}~Melee Weapon Attack: +9 to hit, reach 15 ft., one
target. Hit: 19 (4d6 + 5) bludgeoning damage. The aboleth can choose
instead to deal 0 damage. If the target is a creature it makes a DC 16
Constitution saving throw. On a failure it contracts a disease called
the Sea Change. On a success it is immune to this disease for 24 hours.
While affected by this disease the target has disadvantage on Wisdom
saving throws. After 1 hour the target grows gills it can breathe water
its skin becomes slimy and it begins to suffocate if it goes 12 hours
without being immersed in water for at least 1 hour. This disease can be
removed with a disease-removing spell cast with at least a 4th-level
spell slot and it ends 24 hours after the aboleth dies.

\textbf{Slimy Cloud (1/Day, While Bloodied):}~While underwater the
aboleth exudes a cloud of inky slime in a 30-foot-radius sphere. Each
non-aboleth creature in the area when the cloud appears makes a DC 16
Constitution saving throw. On a failure it takes 44 (8d10) poison damage
and is poisoned for 1 minute. The slime extends around corners and the
area is heavily obscured for 1 minute or until a strong current
dissipates the cloud.

\subsubsection{Legendary Actions}\label{legendary-actions}

\textbf{The aboleth can take 2 legendary actions:}~Only one legendary
action can be used at a time and only at the end of another creatures
turn. It regains spent legendary actions at the start of its turn.

\textbf{Move:}~The aboleth moves up to its swim speed without provoking
opportunity attacks.

\textbf{Telepathic Summon:}~One creature within 90 feet makes a DC 16
Wisdom saving throw. On a failure, it must use its reaction, if
available, to move up to its speed toward the aboleth by the most direct
route that avoids hazards, not avoiding opportunity attacks. This is a
magical charm effect.

\textbf{Baleful Charm (Costs 2 Actions):}~The aboleth targets one
creature within 60 feet that has contracted Sea Change. The target makes
a DC 16 Wisdom saving throw. On a failure, it is magically charmed by
the aboleth until the aboleth dies. The target can repeat this saving
throw every 24 hours and when it takes damage from the aboleth or the
aboleths allies. While charmed in this way, the target can communicate
telepathically with the aboleth over any distance and it follows the
aboleths orders.

\textbf{Soul Drain (Costs 2 Actions):}~One creature charmed by the
aboleth takes 22 (4d10) psychic damage, and the aboleth regains hit
points equal to the damage dealt.

\endgroup

\vfill\break

\begingroup

\makeatletter
\clubpenalty=150
\widowpenalty=150
\displaywidowpenalty=150
\monsterFont
\fontsize{9pt}{10pt}\selectfont
\setlength{\parskip}{4pt}
\makeatletter
\setlist[itemize]{left=1.5em, itemsep=1pt, topsep=6pt, parsep=0pt, partopsep=0pt}
\setlist[enumerate]{left=1.5em, itemsep=2pt, topsep=6pt, parsep=0pt, partopsep=0pt}

\renewcommand{\sectionsize}{\Large}
\renewcommand{\subsectionsize}{\normalsize}
\renewcommand{\subsubsectionsize}{\normalsize}

\titlespacing*{\section}{0pt}{6pt}{4pt}
\titlespacing*{\subsection}{0pt}{6pt}{4pt}
\titlespacing*{\subsubsection}{0pt}{4pt}{4pt}
\titlespacing*{\subsubsubsection}{0pt}{4pt}{4pt}

\titleformat{\section}[block]
  {\stickysubsection\sectionsize\color{sectioncolor}\headerfontbold}
  {}
  {0pt}
  {}
  [\vspace{0pt}\color{sectioncolor}\hrule height 1pt]

\titleformat{\subsection}[block]
  {\stickysubsection\subsectionsize\color{subsectioncolor}\headerfont}
  {}
  {0pt}
  {}
  [\vspace{0pt}\color{sectioncolor}\hrule height 1pt]
\titleformat{\subsubsection}[block]
  {\stickysubsection\subsubsectionsize\color{subsubsectioncolor}\headerfont}
  {}
  {0pt}
  {}
  [\vspace{0pt}\color{sectioncolor}\hrule height 1pt]
% Local override for \tightlist so global version doesn't bleed in
\def\tightlist{%
  \setlength{\itemsep}{0pt}%
  \setlength{\topsep}{4pt}%
  \setlength{\parsep}{0pt}%
  \setlength{\parskip}{0pt}%
  \setlength{\partopsep}{0pt}%
}
\makeatother

\section{Boneclaw, Brine}\label{boneclaw-brine}

\emph{Large Undead, Chaotic Evil}

\setlength{\itemsep}{0pt}

\begin{itemize}
\tightlist
\item
  \textbf{Armor Class:}~16 (natural armor)
\item
  \textbf{Hit Points:}~175 (17d10 + 34)
\item
  \textbf{Speed:}~40 ft.
\item
  \textbf{Initiative}: +3 (13)
\end{itemize}

\begin{center}
{\sffamily\fontsize{8pt}{8pt}\selectfont
\rowcolors{2}{highlightcolor}{white}
\begin{tabular}{llll}
\toprule
\textbf{STAT} & \textbf{SCORE} & \textbf{MOD} & \textbf{SAVE} \\
\midrule
STR & 19 & +4 & +4 \\
DEX & 16 & +3 & +7 \\
CON & 15 & +2 & +6 \\
INT & 13 & +1 & +1 \\
WIS & 15 & +2 & +6 \\
CHA & 9 & \texttt{-1} & \texttt{-1} \\
\bottomrule
\end{tabular}}
\end{center}

\setlength{\itemsep}{0pt}

\begin{itemize}
\tightlist
\item
  \textbf{Skills:} Perception +6, Stealth +7
\item
  \textbf{Resistances}: Cold, Fire, Necrotic, Bludgeoning, Piercing,
  Slashing from nonmagical attacks
\item
  \textbf{Immunities}: Poison; Charmed, Exhaustion, Frightened,
  Paralyzed, Poisoned
\item
  \textbf{Senses}: Darkvision 60 ft.; Passive Perception 16
\item
  \textbf{Languages}: Common plus the main language of its master
\item
  \textbf{CR}~12 (XP 8,400)
\end{itemize}

\subsection{Tactics}\label{tactics}

\setlength{\itemsep}{0pt}

\begin{itemize}
\tightlist
\item
  Ambush while unseen; Use 15-foot reach and grapple/pull
\item
  Shadow Stealth after attack~
\item
  Deadly Reach anyone coming to their aid
\item
  Shadow Jump if enemies get too close
\item
  Disarm anyone holding light: \emph{The attacker makes an attack roll
  contested by the target's Strength (Athletics) check or Dexterity
  (Acrobatics) check. If the attacker wins the contest, the attack
  causes no damage or other ill effect, but the defender drops the
  item.}
\end{itemize}

\subsection{Traits}\label{traits}

\emph{\textbf{Rejuvenation.}}~While its master lives, a destroyed
boneclaw gains a new body in~1d10~hours, with all its hit points. The
new body appears within 1 mile of the boneclaw's master.

\emph{\textbf{Shadow Stealth.}}~While in dim light or darkness, the
boneclaw can take the Hide action as a bonus action.

\emph{\textbf{Reaping Claws.}}~The boneclaw's claws are magical. When it
hits a target with them, the target's hit point maximum is reduced by an
amount equal to the necrotic damage taken. This reduction lasts until
the target finishes a long rest. The target dies if this effect reduces
its hit point maximum to 0.

\subsection{Actions}\label{actions-1}

\emph{\textbf{Multiattack.}}~The boneclaw makes two claw attacks.

\emph{\textbf{Claw.}}~\emph{Melee Weapon Attack:}~+8 to hit, reach 15
ft., one target.~\emph{Hit:}~20 (3d10 + 4) piercing damage plus 9 (2d8)
necrotic damage.~The boneclaw has two claws. While a claw grapples a
target, the claw can attack only that target.

\emph{\textbf{Shadow Jump.}}~If the boneclaw is in dim light or
darkness, each creature of the boneclaw's choice within 5 feet of it
must succeed on a DC~14~Constitution saving throw or take 34 (5d12 + 2)
necrotic damage. The boneclaw then magically teleports up to 60 feet to
an unoccupied space it can see. It can bring one creature it's
grappling, teleporting that creature to an unoccupied space it can see
within 5 feet of its destination. The destination spaces of this
teleportation must be in dim light or darkness.

\subsection{Reactions}\label{reactions}

\textbf{\emph{Deadly Reach.}}~In response to a visible enemy moving into
its reach, the boneclaw makes one claw attack against that enemy. If the
attack hits, the boneclaw can make a second claw attack against the
target.

\endgroup

\vfill\break

\begingroup

\makeatletter
\clubpenalty=150
\widowpenalty=150
\displaywidowpenalty=150
\monsterFont
\fontsize{9pt}{10pt}\selectfont
\setlength{\parskip}{4pt}
\makeatletter
\setlist[itemize]{left=1.5em, itemsep=1pt, topsep=6pt, parsep=0pt, partopsep=0pt}
\setlist[enumerate]{left=1.5em, itemsep=2pt, topsep=6pt, parsep=0pt, partopsep=0pt}

\renewcommand{\sectionsize}{\Large}
\renewcommand{\subsectionsize}{\normalsize}
\renewcommand{\subsubsectionsize}{\normalsize}

\titlespacing*{\section}{0pt}{6pt}{4pt}
\titlespacing*{\subsection}{0pt}{6pt}{4pt}
\titlespacing*{\subsubsection}{0pt}{4pt}{4pt}
\titlespacing*{\subsubsubsection}{0pt}{4pt}{4pt}

\titleformat{\section}[block]
  {\stickysubsection\sectionsize\color{sectioncolor}\headerfontbold}
  {}
  {0pt}
  {}
  [\vspace{0pt}\color{sectioncolor}\hrule height 1pt]

\titleformat{\subsection}[block]
  {\stickysubsection\subsectionsize\color{subsectioncolor}\headerfont}
  {}
  {0pt}
  {}
  [\vspace{0pt}\color{sectioncolor}\hrule height 1pt]
\titleformat{\subsubsection}[block]
  {\stickysubsection\subsubsectionsize\color{subsubsectioncolor}\headerfont}
  {}
  {0pt}
  {}
  [\vspace{0pt}\color{sectioncolor}\hrule height 1pt]
% Local override for \tightlist so global version doesn't bleed in
\def\tightlist{%
  \setlength{\itemsep}{0pt}%
  \setlength{\topsep}{4pt}%
  \setlength{\parsep}{0pt}%
  \setlength{\parskip}{0pt}%
  \setlength{\partopsep}{0pt}%
}
\makeatother

\section{Corpse Flower}\label{corpse-flower}

\emph{Large Plant, Chaotic Evil}

\setlength{\itemsep}{0pt}

\begin{itemize}
\tightlist
\item
  \textbf{Armor Class:}~12
\item
  \textbf{Hit Points:}~168 (maxed)
\item
  \textbf{Speed:}~20 ft., Climb 20 ft.
\item
  \textbf{Initiative}: Acts at 15, 10, 5, 0
\end{itemize}

\begin{center}
{\sffamily\fontsize{8pt}{8pt}\selectfont
\rowcolors{2}{highlightcolor}{white}
\begin{tabular}{llll}
\toprule
\textbf{STAT} & \textbf{SCORE} & \textbf{MOD} & \textbf{SAVE} \\
\midrule
STR & 14 & +2 & — \\
DEX & 14 & +2 & — \\
CON & 16 & +3 & +9 \\
INT & 7 & −2 & — \\
WIS & 15 & +2 & +4 \\
CHA & 3 & −4 & — \\
\bottomrule
\end{tabular}}
\end{center}

\setlength{\itemsep}{0pt}

\begin{itemize}
\tightlist
\item
  \textbf{Resistances}: Cold, Lightning, Necrotic; nonmagical
  Bludgeoning, Piercing, and Slashing
\item
  \textbf{Immunities}: Poison, Poisoned
\item
  \textbf{Senses}: Blindsight 120 ft.
\item
  \textbf{CR}~8
\end{itemize}

\subsection{Tactics}\label{tactics-1}

\setlength{\itemsep}{0pt}

\begin{itemize}
\tightlist
\item
  It will attempt to climb a tree to get out of melee reach but still
  attack with its 10 ft reach
\item
  Digest a corpse every round it's been injured until no more remain
\item
  Continue to attack enemies reduced to 0 HP until they are dead, then
  harvest
\item
  When it is reduced to 50 hp or fewer, it will disengage and then dash
  away, climbing a tree if necessary to escape
\end{itemize}

\subsection{Traits}\label{traits-1}

\emph{\textbf{Stench of Death.}}~Creatures within 10 feet must make a DC
14 Constitution saving throw or be Poisoned until the start of their
next turn.

\emph{\textbf{Regrowth.}}~The corpse flower regains 5 HP at the start of
its turn if it has at least one corpse in its body.

\emph{\textbf{Corpses.}}~The corpse flower holds 1d6 + 3 humanoid
corpses, which it can use to heal or to spawn zombies.

\emph{\textbf{Harvest the Dead.}}~The corpse flower grabs one unsecured
dead humanoid within 10 feet and stuffs it into its body, along with any
equipment it was wearing or carrying. These remains can be used with the
Corpses trait.

\subsection{Actions}\label{actions-2}

\emph{\textbf{Multiattack (Full Turns).}}~The corpse flower makes three
Tentacle attacks and uses either Digest Corpse or Spawn Zombie.

\emph{\textbf{Tentacle.}}~\emph{Melee Attack Roll:}~+5, reach 10
ft.~\emph{Hit:}~6 (2d4 + 2) Bludgeoning damage. The target must succeed
on a DC 14 Constitution saving throw or take 8 (4d4) Poison damage.

\emph{\textbf{Digest Corpse.}}~The corpse flower digests one humanoid
corpse inside it, regaining 10 (2d10) HP. Each creature within 10 feet
takes 4 (2d4) Necrotic damage (\emph{Constitution Saving Throw:}~DC
15,~\emph{Success:}~half damage).

\emph{\textbf{Spawn Zombie.}}~The corpse flower expels a corpse as a
zombie into an adjacent space. Maximum of 6 zombies can be active at
once.

\subsection{Bonus Actions}\label{bonus-actions}

\emph{\textbf{Digest Corpse.}}~The corpse flower digests a humanoid
corpse inside it and regains 11 (2d10) HP. The corpse and its remains
are destroyed; any equipment is expelled into its space.

\emph{\textbf{Animate Corpse.}}~The corpse flower animates a humanoid
corpse inside it as a zombie, which appears in an unoccupied space
within 5 feet. It acts immediately after the corpse flower in the
initiative. It is not under control but acts as an ally. The zombie
emits the same Stench of Death.

\subsection{Reaction}\label{reaction}

\emph{\textbf{Grasping Vines (Recharge 5--6).}}~When a creature misses
the corpse flower with a melee attack, the flower lashes out. The
attacker must make a DC 14 Dexterity saving throw or become Grappled
(escape DC 14).

\endgroup

\vfill\break

\begingroup

\makeatletter
\clubpenalty=150
\widowpenalty=150
\displaywidowpenalty=150
\monsterFont
\fontsize{9pt}{10pt}\selectfont
\setlength{\parskip}{4pt}
\makeatletter
\setlist[itemize]{left=1.5em, itemsep=1pt, topsep=6pt, parsep=0pt, partopsep=0pt}
\setlist[enumerate]{left=1.5em, itemsep=2pt, topsep=6pt, parsep=0pt, partopsep=0pt}

\renewcommand{\sectionsize}{\Large}
\renewcommand{\subsectionsize}{\normalsize}
\renewcommand{\subsubsectionsize}{\normalsize}

\titlespacing*{\section}{0pt}{6pt}{4pt}
\titlespacing*{\subsection}{0pt}{6pt}{4pt}
\titlespacing*{\subsubsection}{0pt}{4pt}{4pt}
\titlespacing*{\subsubsubsection}{0pt}{4pt}{4pt}

\titleformat{\section}[block]
  {\stickysubsection\sectionsize\color{sectioncolor}\headerfontbold}
  {}
  {0pt}
  {}
  [\vspace{0pt}\color{sectioncolor}\hrule height 1pt]

\titleformat{\subsection}[block]
  {\stickysubsection\subsectionsize\color{subsectioncolor}\headerfont}
  {}
  {0pt}
  {}
  [\vspace{0pt}\color{sectioncolor}\hrule height 1pt]
\titleformat{\subsubsection}[block]
  {\stickysubsection\subsubsectionsize\color{subsubsectioncolor}\headerfont}
  {}
  {0pt}
  {}
  [\vspace{0pt}\color{sectioncolor}\hrule height 1pt]
% Local override for \tightlist so global version doesn't bleed in
\def\tightlist{%
  \setlength{\itemsep}{0pt}%
  \setlength{\topsep}{4pt}%
  \setlength{\parsep}{0pt}%
  \setlength{\parskip}{0pt}%
  \setlength{\partopsep}{0pt}%
}
\makeatother

\section{Ghast Gravecaller}\label{ghast-gravecaller}

\emph{Medium Undead, Chaotic Evil}

\setlength{\itemsep}{0pt}

\begin{itemize}
\tightlist
\item
  \textbf{Armor Class:}~16
\item
  \textbf{Hit Points:}~97 (15d8 + 30)
\item
  \textbf{Speed:}~30 ft.
\item
  \textbf{Initiative}: +3 (13)
\end{itemize}

\begin{center}
{\sffamily\fontsize{8pt}{8pt}\selectfont
\rowcolors{2}{highlightcolor}{white}
\begin{tabular}{llll}
\toprule
\textbf{STAT} & \textbf{SCORE} & \textbf{MOD} & \textbf{SAVE} \\
\midrule
STR & 16 & +3 & +3 \\
DEX & 17 & +3 & +3 \\
CON & 14 & +2 & +5 \\
INT & 18 & +4 & +4 \\
WIS & 14 & +2 & +5 \\
CHA & 8 & −1 & −1 \\
\bottomrule
\end{tabular}}
\end{center}

\setlength{\itemsep}{0pt}

\begin{itemize}
\tightlist
\item
  \textbf{Immunities}: Necrotic, Poison; Charmed, Exhaustion, Poisoned
\item
  \textbf{Resistance:} Fire
\item
  \textbf{Senses}: Darkvision 120 ft.; Passive Perception 12
\item
  \textbf{Languages}: Abyssal, Common
\item
  \textbf{CR}~6 (XP 2,300; PB +3)
\end{itemize}

\subsection{Tactics}\label{tactics-2}

\setlength{\itemsep}{0pt}

\begin{itemize}
\tightlist
\item
  Claw attack intending to paralyze, then Swamp Grasp into the muck
\item
  Drag paralyzed creatures away for a snack, continuing to use claws
\item
  Smart enough to avoid PCs immune or resistant to poison
\item
  Dash toward opponents using ranged weapons against them
\item
  Retreats when \textless= 14 HP
\end{itemize}

\subsection{Traits}\label{traits-2}

\emph{\textbf{Stench.}}~\emph{Constitution Saving Throw}: DC 13, any
creature that starts its turn in a 5-foot Emanation originating from the
ghast.~\emph{Failure:}~The target has the Poisoned condition until the
start of its next turn.~\emph{Success:}~The target is immune to this
ghast's Stench for 24 hours.

\subsection{Actions}\label{actions-3}

\emph{\textbf{Multiattack.}}~The ghast makes two Horrific Necrosis
attacks. It can replace one attack with a Claw attack.

\emph{\textbf{Claw.}}~\emph{Melee Attack Roll}: +6, reach 5
ft.~\emph{Hit:}~13 (3d6 + 3) Slashing damage. If the target isn't an
Undead, it has the Paralyzed condition until the end of its next turn.

\emph{\textbf{Horrific Necrosis.}}~\emph{Melee or Ranged Attack Roll}:
+7, reach 5 ft. or range 120 ft.~\emph{Hit:}~15 (2d10 + 4) Necrotic
damage, and the target has the Frightened condition until the end of its
next turn.

\emph{\textbf{Spellcasting.}}~The ghast casts one of the following
spells, requiring no Material components and using Intelligence as the
spellcasting ability:\\
\textbf{At Will:}~\emph{Speak with Dead},~\emph{Thaumaturgy}

\subsection{Bonus Action}\label{bonus-action}

\emph{\textbf{Swamp Grasp (Recharge 5--6).}}~When the Ghast Gravecaller
hits a Medium or smaller creature with a melee attack, it can attempt to
drag the target 5 feet into the water. The target must succeed on a DC
13 Strength saving throw or be knocked prone and pulled into the muck,
where it is~\textbf{Restrained}~until it or an ally uses an action to
pull it free (DC 13 Strength).

\endgroup

\vfill\break

\begingroup

\makeatletter
\clubpenalty=150
\widowpenalty=150
\displaywidowpenalty=150
\monsterFont
\fontsize{9pt}{10pt}\selectfont
\setlength{\parskip}{4pt}
\makeatletter
\setlist[itemize]{left=1.5em, itemsep=1pt, topsep=6pt, parsep=0pt, partopsep=0pt}
\setlist[enumerate]{left=1.5em, itemsep=2pt, topsep=6pt, parsep=0pt, partopsep=0pt}

\renewcommand{\sectionsize}{\Large}
\renewcommand{\subsectionsize}{\normalsize}
\renewcommand{\subsubsectionsize}{\normalsize}

\titlespacing*{\section}{0pt}{6pt}{4pt}
\titlespacing*{\subsection}{0pt}{6pt}{4pt}
\titlespacing*{\subsubsection}{0pt}{4pt}{4pt}
\titlespacing*{\subsubsubsection}{0pt}{4pt}{4pt}

\titleformat{\section}[block]
  {\stickysubsection\sectionsize\color{sectioncolor}\headerfontbold}
  {}
  {0pt}
  {}
  [\vspace{0pt}\color{sectioncolor}\hrule height 1pt]

\titleformat{\subsection}[block]
  {\stickysubsection\subsectionsize\color{subsectioncolor}\headerfont}
  {}
  {0pt}
  {}
  [\vspace{0pt}\color{sectioncolor}\hrule height 1pt]
\titleformat{\subsubsection}[block]
  {\stickysubsection\subsubsectionsize\color{subsubsectioncolor}\headerfont}
  {}
  {0pt}
  {}
  [\vspace{0pt}\color{sectioncolor}\hrule height 1pt]
% Local override for \tightlist so global version doesn't bleed in
\def\tightlist{%
  \setlength{\itemsep}{0pt}%
  \setlength{\topsep}{4pt}%
  \setlength{\parsep}{0pt}%
  \setlength{\parskip}{0pt}%
  \setlength{\partopsep}{0pt}%
}
\makeatother

\section{Giant Crab}\label{giant-crab}

\emph{Medium Beast, Unaligned}

\setlength{\itemsep}{0pt}

\begin{itemize}
\tightlist
\item
  \textbf{Armor Class:}~15
\item
  \textbf{Hit Points:}~24 (3d8) (Maximum)
\item
  \textbf{Speed:}~30 ft., Swim 30 ft.
\item
  \textbf{Initiative}: +1 (11)
\end{itemize}

\begin{center}
{\sffamily\fontsize{8pt}{8pt}\selectfont
\rowcolors{2}{highlightcolor}{white}
\begin{tabular}{llll}
\toprule
\textbf{STAT} & \textbf{SCORE} & \textbf{MOD} & \textbf{SAVE} \\
\midrule
STR & 13 & +1 & +1 \\
DEX & 13 & +1 & +1 \\
CON & 11 & +0 & +0 \\
INT & 1 & \texttt{-5} & \texttt{-5} \\
WIS & 9 & \texttt{-1} & \texttt{-1} \\
CHA & 3 & \texttt{-4} & \texttt{-4} \\
\bottomrule
\end{tabular}}
\end{center}

\setlength{\itemsep}{0pt}

\begin{itemize}
\tightlist
\item
  \textbf{Skills}: Stealth +3
\item
  \textbf{Senses}: blindsight 30 ft.; Passive Perception 9
\item
  \textbf{CR}~1/8 (XP 25; PB +2)
\end{itemize}

\subsection{Traits}\label{traits-3}

\emph{\textbf{Amphibious.}}~The crab can breathe air and water.

\subsection{Actions}\label{actions-4}

\emph{\textbf{Claw.}}~\emph{Melee Attack Roll:}~+3, reach 5 ft. 4 (1d6 +
1) Bludgeoning damage. If the target is a Medium or smaller creature, it
has the Grappled condition (escape DC 11) from one of two claws.

\endgroup

\vfill\break

\begingroup

\makeatletter
\clubpenalty=150
\widowpenalty=150
\displaywidowpenalty=150
\monsterFont
\fontsize{9pt}{10pt}\selectfont
\setlength{\parskip}{4pt}
\makeatletter
\setlist[itemize]{left=1.5em, itemsep=1pt, topsep=6pt, parsep=0pt, partopsep=0pt}
\setlist[enumerate]{left=1.5em, itemsep=2pt, topsep=6pt, parsep=0pt, partopsep=0pt}

\renewcommand{\sectionsize}{\Large}
\renewcommand{\subsectionsize}{\normalsize}
\renewcommand{\subsubsectionsize}{\normalsize}

\titlespacing*{\section}{0pt}{6pt}{4pt}
\titlespacing*{\subsection}{0pt}{6pt}{4pt}
\titlespacing*{\subsubsection}{0pt}{4pt}{4pt}
\titlespacing*{\subsubsubsection}{0pt}{4pt}{4pt}

\titleformat{\section}[block]
  {\stickysubsection\sectionsize\color{sectioncolor}\headerfontbold}
  {}
  {0pt}
  {}
  [\vspace{0pt}\color{sectioncolor}\hrule height 1pt]

\titleformat{\subsection}[block]
  {\stickysubsection\subsectionsize\color{subsectioncolor}\headerfont}
  {}
  {0pt}
  {}
  [\vspace{0pt}\color{sectioncolor}\hrule height 1pt]
\titleformat{\subsubsection}[block]
  {\stickysubsection\subsubsectionsize\color{subsubsectioncolor}\headerfont}
  {}
  {0pt}
  {}
  [\vspace{0pt}\color{sectioncolor}\hrule height 1pt]
% Local override for \tightlist so global version doesn't bleed in
\def\tightlist{%
  \setlength{\itemsep}{0pt}%
  \setlength{\topsep}{4pt}%
  \setlength{\parsep}{0pt}%
  \setlength{\parskip}{0pt}%
  \setlength{\partopsep}{0pt}%
}
\makeatother

\section{Giant Shadow Eel}\label{giant-shadow-eel}

\emph{Huge Beast, Unaligned}

\setlength{\itemsep}{0pt}

\begin{itemize}
\tightlist
\item
  \textbf{Armor Class:}~12
\item
  \textbf{Hit Points:}~104 (8d12 + 8) (Maximum)
\item
  \textbf{Speed:}~30 ft., Swim 30 ft.
\item
  \textbf{Initiative}: +2 (12)
\end{itemize}

\begin{center}
{\sffamily\fontsize{8pt}{8pt}\selectfont
\rowcolors{2}{highlightcolor}{white}
\begin{tabular}{llll}
\toprule
\textbf{STAT} & \textbf{SCORE} & \textbf{MOD} & \textbf{SAVE} \\
\midrule
STR & 19 & +4 & +4 \\
DEX & 14 & +2 & +2 \\
CON & 12 & +1 & +1 \\
INT & 1 & \texttt{-5} & \texttt{-5} \\
WIS & 10 & +0 & +0 \\
CHA & 3 & \texttt{-4} & \texttt{-4} \\
\bottomrule
\end{tabular}}
\end{center}

\setlength{\itemsep}{0pt}

\begin{itemize}
\tightlist
\item
  \textbf{Skills}: Perception +2
\item
  \textbf{Senses}: blindsight 10 ft.; Passive Perception 12
\item
  \textbf{CR}~2 (XP 450; PB +2)
\end{itemize}

\subsection{Actions}\label{actions-5}

\emph{\textbf{Multiattack.}}~The eel makes one Bite attack and uses
Constrict.

\emph{\textbf{Bite.}}~\emph{Melee Attack Roll:}~+6, reach 10 ft. 11 (2d6
+ 4) Piercing damage.

\emph{\textbf{Constrict.}}~\emph{Strength Saving Throw}: DC 14, one
Large or smaller creature the eel can see within 10
feet.~\emph{Failure:}~13 (2d8 + 4) Bludgeoning damage, and the target
has the Grappled condition (escape DC 14).

\subsection{Traits}\label{traits-4}

\emph{\textbf{Amphibious.}}~The eel can breathe air and water.

\endgroup

\vfill\break

\begingroup

\makeatletter
\clubpenalty=150
\widowpenalty=150
\displaywidowpenalty=150
\monsterFont
\fontsize{9pt}{10pt}\selectfont
\setlength{\parskip}{4pt}
\makeatletter
\setlist[itemize]{left=1.5em, itemsep=1pt, topsep=6pt, parsep=0pt, partopsep=0pt}
\setlist[enumerate]{left=1.5em, itemsep=2pt, topsep=6pt, parsep=0pt, partopsep=0pt}

\renewcommand{\sectionsize}{\Large}
\renewcommand{\subsectionsize}{\normalsize}
\renewcommand{\subsubsectionsize}{\normalsize}

\titlespacing*{\section}{0pt}{6pt}{4pt}
\titlespacing*{\subsection}{0pt}{6pt}{4pt}
\titlespacing*{\subsubsection}{0pt}{4pt}{4pt}
\titlespacing*{\subsubsubsection}{0pt}{4pt}{4pt}

\titleformat{\section}[block]
  {\stickysubsection\sectionsize\color{sectioncolor}\headerfontbold}
  {}
  {0pt}
  {}
  [\vspace{0pt}\color{sectioncolor}\hrule height 1pt]

\titleformat{\subsection}[block]
  {\stickysubsection\subsectionsize\color{subsectioncolor}\headerfont}
  {}
  {0pt}
  {}
  [\vspace{0pt}\color{sectioncolor}\hrule height 1pt]
\titleformat{\subsubsection}[block]
  {\stickysubsection\subsubsectionsize\color{subsubsectioncolor}\headerfont}
  {}
  {0pt}
  {}
  [\vspace{0pt}\color{sectioncolor}\hrule height 1pt]
% Local override for \tightlist so global version doesn't bleed in
\def\tightlist{%
  \setlength{\itemsep}{0pt}%
  \setlength{\topsep}{4pt}%
  \setlength{\parsep}{0pt}%
  \setlength{\parskip}{0pt}%
  \setlength{\partopsep}{0pt}%
}
\makeatother

\section{Pirate}\label{pirate}

\emph{Medium or Small Humanoid, Neutral}

\setlength{\itemsep}{0pt}

\begin{itemize}
\tightlist
\item
  \textbf{Armor Class:}~14
\item
  \textbf{Hit Points:}~33 (6d8 + 6)
\item
  \textbf{Speed:}~30 ft.
\item
  \textbf{Initiative}: +5 (15)
\end{itemize}

\begin{center}
{\sffamily\fontsize{8pt}{8pt}\selectfont
\rowcolors{2}{highlightcolor}{white}
\begin{tabular}{llll}
\toprule
\textbf{STAT} & \textbf{SCORE} & \textbf{MOD} & \textbf{SAVE} \\
\midrule
STR & 10 & +0 & +0 \\
DEX & 16 & +3 & +5 \\
CON & 12 & +1 & +1 \\
INT & 8 & \texttt{-1} & \texttt{-1} \\
WIS & 12 & +1 & +1 \\
CHA & 14 & +2 & +4 \\
\bottomrule
\end{tabular}}
\end{center}

\setlength{\itemsep}{0pt}

\begin{itemize}
\tightlist
\item
  \textbf{Gear}~Dagger x 6, Leather Armor
\item
  \textbf{Senses}: Passive Perception 11
\item
  \textbf{Languages}: Common plus one other language
\item
  \textbf{CR}~1 (XP 200; PB +2)
\end{itemize}

\subsection{Actions}\label{actions-6}

\emph{\textbf{Multiattack.}}~The pirate makes two Dagger attacks or one
Crossbow attack. It can replace one attack with a use of Enthralling
Panache.

\emph{\textbf{Dagger.}}~\emph{Melee or Ranged Attack Roll:}~+5, reach 5
ft. or range 20/60 ft. 5 (1d4 + 3) Piercing damage.

\emph{\textbf{Heavy Crossbow.} Ranged Attack Roll:} +5, range 100
ft./400 ft. (1d10 + 5) Piercing damage.

\emph{\textbf{Enthralling Panache.}}~\emph{Wisdom Saving Throw}: DC 12,
one creature the pirate can see within 30 feet.~\emph{Failure:}~The
target has the Charmed condition until the start of the pirate's next
turn.

\endgroup

\vfill\break

\begingroup

\makeatletter
\clubpenalty=150
\widowpenalty=150
\displaywidowpenalty=150
\monsterFont
\fontsize{9pt}{10pt}\selectfont
\setlength{\parskip}{4pt}
\makeatletter
\setlist[itemize]{left=1.5em, itemsep=1pt, topsep=6pt, parsep=0pt, partopsep=0pt}
\setlist[enumerate]{left=1.5em, itemsep=2pt, topsep=6pt, parsep=0pt, partopsep=0pt}

\renewcommand{\sectionsize}{\Large}
\renewcommand{\subsectionsize}{\normalsize}
\renewcommand{\subsubsectionsize}{\normalsize}

\titlespacing*{\section}{0pt}{6pt}{4pt}
\titlespacing*{\subsection}{0pt}{6pt}{4pt}
\titlespacing*{\subsubsection}{0pt}{4pt}{4pt}
\titlespacing*{\subsubsubsection}{0pt}{4pt}{4pt}

\titleformat{\section}[block]
  {\stickysubsection\sectionsize\color{sectioncolor}\headerfontbold}
  {}
  {0pt}
  {}
  [\vspace{0pt}\color{sectioncolor}\hrule height 1pt]

\titleformat{\subsection}[block]
  {\stickysubsection\subsectionsize\color{subsectioncolor}\headerfont}
  {}
  {0pt}
  {}
  [\vspace{0pt}\color{sectioncolor}\hrule height 1pt]
\titleformat{\subsubsection}[block]
  {\stickysubsection\subsubsectionsize\color{subsubsectioncolor}\headerfont}
  {}
  {0pt}
  {}
  [\vspace{0pt}\color{sectioncolor}\hrule height 1pt]
% Local override for \tightlist so global version doesn't bleed in
\def\tightlist{%
  \setlength{\itemsep}{0pt}%
  \setlength{\topsep}{4pt}%
  \setlength{\parsep}{0pt}%
  \setlength{\parskip}{0pt}%
  \setlength{\partopsep}{0pt}%
}
\makeatother

\section{Pirate Captain}\label{pirate-captain}

\emph{Medium or Small Humanoid, Neutral}

\setlength{\itemsep}{0pt}

\begin{itemize}
\tightlist
\item
  \textbf{Armor Class:}~17
\item
  \textbf{Hit Points:}~84 (13d8 + 26)
\item
  \textbf{Speed:}~30 ft.
\item
  \textbf{Initiative}: +7 (17)
\end{itemize}

\begin{center}
{\sffamily\fontsize{8pt}{8pt}\selectfont
\rowcolors{2}{highlightcolor}{white}
\begin{tabular}{llll}
\toprule
\textbf{STAT} & \textbf{SCORE} & \textbf{MOD} & \textbf{SAVE} \\
\midrule
STR & 10 & +0 & +3 \\
DEX & 18 & +4 & +7 \\
CON & 14 & +2 & +2 \\
INT & 10 & +0 & +0 \\
WIS & 14 & +2 & +5 \\
CHA & 17 & +3 & +6 \\
\bottomrule
\end{tabular}}
\end{center}

\setlength{\itemsep}{0pt}

\begin{itemize}
\tightlist
\item
  \textbf{Skills}: Acrobatics +7, Perception +5
\item
  \textbf{Gear}~Pistol, Rapier
\item
  \textbf{Senses}: Passive Perception 15
\item
  \textbf{Languages}: Common plus one other language
\item
  \textbf{CR}~6 (XP 2,300; PB +3)
\end{itemize}

\subsection{Actions}\label{actions-7}

\emph{\textbf{Multiattack.}}~The pirate makes three attacks, using
Rapier or Pistol in any combination.

\emph{\textbf{Rapier.}}~\emph{Melee Attack Roll:}~+7, reach 5 ft. 13
(2d8 + 4) Piercing damage, and the pirate has Advantage on the next
attack roll it makes before the end of this turn.

\emph{\textbf{Light Crossbow.}}~\emph{Ranged Attack Roll:}~+7, range
30/90 ft. 15 (2d10 + 4) Piercing damage.

\subsection{Bonus Actions}\label{bonus-actions-1}

\emph{\textbf{Captain's Charm.}}~\emph{Wisdom Saving Throw}: DC 14, one
creature the pirate can see within 30 feet.~\emph{Failure:}~The target
has the Charmed condition until the start of the pirate's next turn.

\endgroup

\vfill\break

\begingroup

\makeatletter
\clubpenalty=150
\widowpenalty=150
\displaywidowpenalty=150
\monsterFont
\fontsize{9pt}{10pt}\selectfont
\setlength{\parskip}{4pt}
\makeatletter
\setlist[itemize]{left=1.5em, itemsep=1pt, topsep=6pt, parsep=0pt, partopsep=0pt}
\setlist[enumerate]{left=1.5em, itemsep=2pt, topsep=6pt, parsep=0pt, partopsep=0pt}

\renewcommand{\sectionsize}{\Large}
\renewcommand{\subsectionsize}{\normalsize}
\renewcommand{\subsubsectionsize}{\normalsize}

\titlespacing*{\section}{0pt}{6pt}{4pt}
\titlespacing*{\subsection}{0pt}{6pt}{4pt}
\titlespacing*{\subsubsection}{0pt}{4pt}{4pt}
\titlespacing*{\subsubsubsection}{0pt}{4pt}{4pt}

\titleformat{\section}[block]
  {\stickysubsection\sectionsize\color{sectioncolor}\headerfontbold}
  {}
  {0pt}
  {}
  [\vspace{0pt}\color{sectioncolor}\hrule height 1pt]

\titleformat{\subsection}[block]
  {\stickysubsection\subsectionsize\color{subsectioncolor}\headerfont}
  {}
  {0pt}
  {}
  [\vspace{0pt}\color{sectioncolor}\hrule height 1pt]
\titleformat{\subsubsection}[block]
  {\stickysubsection\subsubsectionsize\color{subsubsectioncolor}\headerfont}
  {}
  {0pt}
  {}
  [\vspace{0pt}\color{sectioncolor}\hrule height 1pt]
% Local override for \tightlist so global version doesn't bleed in
\def\tightlist{%
  \setlength{\itemsep}{0pt}%
  \setlength{\topsep}{4pt}%
  \setlength{\parsep}{0pt}%
  \setlength{\parskip}{0pt}%
  \setlength{\partopsep}{0pt}%
}
\makeatother

\section{Wight, Drowned}\label{wight-drowned}

\emph{Medium}~\emph{Undead}

\setlength{\itemsep}{0pt}

\begin{itemize}
\tightlist
\item
  \textbf{Armor Class:}~15
\item
  \textbf{Hit Points:}~45 (6d8+18)
\item
  \textbf{Speed:}~walk 30 ft.
\item
  \textbf{Challenge Rating:}~3 (700 XP)
\item
  \textbf{Source:}~A5e Monstrous Menagerie, page 423
\end{itemize}

\begin{center}
{\sffamily\fontsize{8pt}{8pt}\selectfont
\rowcolors{2}{highlightcolor}{white}
\begin{tabular}{llllll}
\toprule
\textbf{STR} & \textbf{DEX} & \textbf{CON} & \textbf{INT} & \textbf{WIS} & \textbf{CHA} \\
\midrule
14 (+2) & 14 (+2) & 16 (+3) & 10 (+0) & 12 (+1) & 14 (+2) \\
\bottomrule
\end{tabular}}
\end{center}

\setlength{\itemsep}{0pt}

\begin{itemize}
\tightlist
\item
  \textbf{Skills:}~perception +3, stealth +4
\item
  \textbf{Damage Resistances:}~cold, fire, necrotic; damage from
  nonmagical, non-silvered weapons
\item
  \textbf{Senses:}~darkvision 60 ft., passive Perception 13
\item
  \textbf{Languages:}~the languages it knew in life
\end{itemize}

\subsubsection{Tactics}\label{tactics-3}

\setlength{\itemsep}{0pt}

\begin{itemize}
\tightlist
\item
  Move with Stealth then attack with surprise
\item
  Use longbow at range
\item
  Use Life Drain if target is reduced HP
\end{itemize}

\subsubsection{Special Abilities}\label{special-abilities-1}

\textbf{\emph{Cold Aura.}}~A creature that starts its turn grappled by
the wight, touches it, or hits it with a melee attack while within 5
feet takes 3 (1d6) cold damage. A creature can take this damage only
once per turn. If the wight has been subjected to fire damage since its
last turn, this trait doesnt function.

\emph{\textbf{Sunlight Sensitivity.}}~While in sunlight, the wight has
disadvantage on attack rolls, as well as on Perception checks that rely
on sight.

\textbf{\emph{Undead Nature.}}~A wight doesnt require air, sustenance,
or sleep.

\subsubsection{Actions}\label{actions-8}

\emph{\textbf{Multiattack.}}~The wight makes two attacks, using
Longsword or Longbow in any combination. It can replace one attack with
a use of Life Drain.

\emph{\textbf{Longsword.}}~Melee Weapon Attack: +4 to hit, reach 5 ft.,
one target. Hit: 6 (1d8 + 2) slashing damage plus 3 (1d6) cold damage.

\textbf{\emph{Seize.}}~Melee Weapon Attack: +4 to hit, reach 5 ft., one
target. Hit: 3 (1d6) cold damage and the target is grappled (escape DC
12). Until this grapple ends the target is restrained and the only
attack the wight can make is Life Drain against the grappled target.

\textbf{\emph{Longbow.}}~Ranged Weapon Attack: +4 to hit, range 150/600
ft., one target. Hit: 6 (1d8 + 2) piercing damage plus 3 (1d6) cold
damage.

\subsubsection{Bonus Actions}\label{bonus-actions-2}

\emph{\textbf{Life Drain.}}~Melee Weapon Attack: +4 to hit, reach 5 ft.,
one creature. Hit: 6 (1d8 + 2) necrotic damage, and the target makes a
DC 13 Constitution saving throw. On a failure, the targets hit point
maximum is reduced by an amount equal to the necrotic damage dealt. The
reduction lasts until the target finishes a long rest. A humanoid or
beast reduced to 0 hit points by this attack dies. Its corpse rises 24
hours later as a zombie under the wights control.

\endgroup

\vfill\break

\begingroup

\makeatletter
\clubpenalty=150
\widowpenalty=150
\displaywidowpenalty=150
\monsterFont
\fontsize{9pt}{10pt}\selectfont
\setlength{\parskip}{4pt}
\makeatletter
\setlist[itemize]{left=1.5em, itemsep=1pt, topsep=6pt, parsep=0pt, partopsep=0pt}
\setlist[enumerate]{left=1.5em, itemsep=2pt, topsep=6pt, parsep=0pt, partopsep=0pt}

\renewcommand{\sectionsize}{\Large}
\renewcommand{\subsectionsize}{\normalsize}
\renewcommand{\subsubsectionsize}{\normalsize}

\titlespacing*{\section}{0pt}{6pt}{4pt}
\titlespacing*{\subsection}{0pt}{6pt}{4pt}
\titlespacing*{\subsubsection}{0pt}{4pt}{4pt}
\titlespacing*{\subsubsubsection}{0pt}{4pt}{4pt}

\titleformat{\section}[block]
  {\stickysubsection\sectionsize\color{sectioncolor}\headerfontbold}
  {}
  {0pt}
  {}
  [\vspace{0pt}\color{sectioncolor}\hrule height 1pt]

\titleformat{\subsection}[block]
  {\stickysubsection\subsectionsize\color{subsectioncolor}\headerfont}
  {}
  {0pt}
  {}
  [\vspace{0pt}\color{sectioncolor}\hrule height 1pt]
\titleformat{\subsubsection}[block]
  {\stickysubsection\subsubsectionsize\color{subsubsectioncolor}\headerfont}
  {}
  {0pt}
  {}
  [\vspace{0pt}\color{sectioncolor}\hrule height 1pt]
% Local override for \tightlist so global version doesn't bleed in
\def\tightlist{%
  \setlength{\itemsep}{0pt}%
  \setlength{\topsep}{4pt}%
  \setlength{\parsep}{0pt}%
  \setlength{\parskip}{0pt}%
  \setlength{\partopsep}{0pt}%
}
\makeatother

\section{Zombie, Cavefish}\label{zombie-cavefish}

\emph{Medium}~\emph{Undead}~\emph{neutral evil}

\setlength{\itemsep}{0pt}

\begin{itemize}
\tightlist
\item
  \textbf{Armor Class:}~10
\item
  \textbf{Hit Points:}~37 (5d8+15)
\item
  \textbf{Speed:}~swim 40 ft. walk 20 ft.
\item
  \textbf{Challenge Rating:}~1/2 (100 XP)
\item
  \textbf{Source:}~Tome of Beasts 2, page 384
\end{itemize}

\begin{center}
{\sffamily\fontsize{8pt}{8pt}\selectfont
\rowcolors{2}{highlightcolor}{white}
\begin{tabular}{llllll}
\toprule
\textbf{STR} & \textbf{DEX} & \textbf{CON} & \textbf{INT} & \textbf{WIS} & \textbf{CHA} \\
\midrule
15 (+2) & 10 (+0) & 16 (+3) & 3 (\texttt{-4}) & 6 (\texttt{-2}) & 3 (\texttt{-4}) \\
\bottomrule
\end{tabular}}
\end{center}

\setlength{\itemsep}{0pt}

\begin{itemize}
\tightlist
\item
  \textbf{Saving Throws}: Wis +0
\item
  \textbf{Resistance:} fire
\item
  \textbf{Damage Immunities:}~poison
\item
  \textbf{Condition Immunities:}~poisoned
\item
  \textbf{Senses:}~darkvision 60 ft., passive Perception 8
\item
  \textbf{Languages:}~understands the languages it knew in life but
  can't speak
\end{itemize}

\subsubsection{Tactics}\label{tactics-4}

\setlength{\itemsep}{0pt}

\begin{itemize}
\tightlist
\item
  Most direct route to opponent, form a barrier to allow wights to
  attack from distance
\item
  Don't forget Undead Fortitude
\end{itemize}

\subsubsection{Special Abilities}\label{special-abilities-2}

\textbf{\emph{Stench.}}~Any creature that starts its turn within 5 feet
of the cavefish zombie must succeed on a DC 10 Constitution saving throw
or be poisoned until the start of its next turn. On a successful saving
throw, the creature is immune to the zombie's Stench for 24 hours.

\textbf{\emph{Undead Fortitude.}}~If damage reduces the cavefish zombie
to 0 hp, it must make a Constitution saving throw with a DC of 5 + the
damage taken, unless the damage is radiant or from a critical hit. On a
success, the zombie drops to 1 hp instead.

\subsubsection{Actions}\label{actions-9}

\textbf{\emph{Slam.}}~Melee Weapon Attack: +4 to hit, reach 5 ft., one
creature. Hit: 5 (1d6 + 2) bludgeoning damage.

\subsubsection{About}\label{about}

This creature looks like a bloated, wet corpse. Its fingers and toes are
webbed, and slick, fleshy fins run down its spine and legs, poking
through stretches of dead flesh. An overpowering stench of rot surrounds
it.

\emph{\textbf{Aquatic Adaptations}.} The cavefish zombie is an unusual
type of undead that occurs when dark magic permeates a lightless, watery
environment, such as in an underground lake or the depths of the ocean.
Rather than retain the bodily form it possessed in life, the creature's
skin sloughs off from parts of its body as aquatic features burst
through its flesh. Its fingers and toes become webbed, and fins form on
its back and down its legs.\\
\emph{\textbf{Decay}.} The cavefish zombie's dead tissue holds water,
causing it to look bloated and loose and afflicting it with a persistent
rot.~This rot results in a horrific odor, which follows them whether
they are in water or on land.\\
\emph{\textbf{Undead Nature}.} A cavefish zombie doesn't require air,
food, drink, or sleep.

\endgroup

\end{document}